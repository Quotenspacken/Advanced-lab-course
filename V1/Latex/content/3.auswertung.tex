\section{Auswertung}\label{sec:aus}
Im folgenden Kapitel werden die Fehlerrechnungen und Ausgleichsrechnungen für die Auswertung der Messdaten mit \textit{numpy}~\cite{numpy} und \textit{scipy}~\cite{scipy} durchgeführt.

\subsection{Kalibrierung des Multikanalanalysators}
Für die Kalibrierung des Multikanalanalysators werden die zeitlichen Abstände am Doppelimpulsgenerator in $\SI{0.5}{\micro\second}$-Schritten von $\SI{1}{\micro\second}$ bis $\SI{5}{\micro\second}$ variiert.
Die zu den am Doppelimpulsgenerator eingestellten zeitlichen Abständen $\Delta t$ erhaltenen Kanäle $K$ sind in \autoref{tab:kal} zu sehen.
\begin{table}[h]
    \centering
    \caption{Am Doppelimpulsgenerator eingestellte Verzögerungen $\Delta t$ und dazu erhaltene Kanäle.}
  \label{tab:kal}
      \begin{tabular}{S S}
        \toprule
        {$\Delta t[\si{\micro\second}]$} & {Kanal}\\
        \midrule
1 & 39 \\
1.5 & 62 \\
2 & 85 \\
2.5 & 108 \\
3 & 131 \\
3.5 & 154 \\
4 & 177 \\
4.5 & 200 \\
5 & 223 \\
        \bottomrule
      \end{tabular}
         \end{table}
\noindent
Diese Messwerte zu der Kalibrierung sind in \autoref{fig:kal} dargestellt. Zusätzlich ist eine Ausgleichsgerade zu sehen, die an die Messwerte angepasst wurde.
\begin{figure}[h!]
    \centering
   % \includegraphics[scale= 0.7]{Abbildungen/kalibrierung.pdf}
    \includegraphics[width=0.7\textwidth]{Abbildungen/kalibrierung.pdf}
    \caption{Darstellung der am Doppelimpulsgenerator eingestellten Verzögerungen $\Delta t$ und dazu erhaltenen Kanalnummern mit Ausgleichsgeraden.}
    \label{fig:kal}
  \end{figure}
\noindent
Die Gesamtanzahl der Kanäle beträgt $512$ entsprechend der Darstellung in \autoref{fig:kal}.
Die ermittelten Regressionsparameter der eingezeichneten Ausgleichsgeraden zur Kalibrierung lauten
\begin{align*}
   a &= \SI{21.74 \pm 0.0}{\nano\second} \\ 
   b &= \SI{152.17 \pm 0}{\nano\second}.
\end{align*}


\subsection{Kalibrierung der Verzögerungsleitung}
Für die Optimierung der Verzögerungszeit der Delays nach den Photomultipliern wird die Verzögerung variiert und jeweils die entsprechende Zählrate gemessen. Die Verzögerungszeiten werden dabei in $\SI{1}{\nano\second}$-Schritten von $\SI{-20}{\nano\second}$ bis $\SI{20}{\nano\second}$ verändert.
Die so gemessenen Werte sind in \autoref{tab:verz} aufgelistet.
\begin{table}[h]
    \centering
      \caption{Gemessene Impulszahl bei Variation der Verzögerung an
den Delays nach den PMTs.}
      \label{tab:verz}
      \begin{tabular}{S S}
        \toprule
        {$\Delta t[\si{\nano\second}]$} & {$N[\si{\per\second}]$}\\
        \midrule
        -20 & 19 \\
-19 & 22 \\
-18 & 20 \\
-17 & 27 \\
-16 & 57 \\
-15 & 27 \\
-13.5 & 66 \\
-13 & 97 \\
-12 & 154 \\
-11 & 270 \\
-10 & 318 \\
-9 & 446 \\
-8 & 533 \\
-7 & 319 \\
-6 & 660 \\
-5 & 842 \\
-4 & 759 \\
-3 & 913 \\
-2 & 1163 \\
-1 & 1191 \\
0 & 1222 \\
1 & 1183 \\
2 & 1224 \\
3 & 1132 \\
4 & 987 \\
5 & 844 \\
6 & 757 \\
7 & 648 \\
8 & 481 \\
9 & 395 \\
10 & 289 \\
11 & 206 \\
12 & 143 \\
13 & 99 \\
14 & 57 \\ 
15 &36 \\ 
16&10 \\ 
17&11 \\ 
18&17 \\ 
19&7 \\ 
20&8 \\ 
        \bottomrule
      \end{tabular}
    \end{table}
\noindent
Diese Messdaten sind in \autoref{fig:verz} dargestellt. Zusätzlich werden an die Daten links und rechts Ausgleichsgeraden und mittig ein Plateau angepasst. Diese Geraden und das Plateau sind ebenfalls in \autoref{fig:verz} zu sehen. Außerdem wurde der halbe Plateauwert eingezeichnet.
\begin{figure}[h]
    \centering
    \includegraphics[width=0.7\textwidth]{Abbildungen/verz.pdf}
    \caption{Gemessene Impulszahl N bei Variation der Verzögerung $\Delta t$ an
den Delays nach den PMTs mit Ausgleichsgeraden und Plateau.}
    \label{fig:verz}
  \end{figure}
\noindent
Aus den Ausgleichsrechnungen zu diesen Messwerten haben sich folgende Parameter ergeben:
\begin{align*}
  a_\text{links} &= 59.57 \pm 5.81\\
  b_\text{links}&= 1006.95 \pm 71.18
\end{align*}

\begin{align*}
  c_\text{Plateau} &= 1185.83 \pm 14.40
\end{align*}

\begin{align*}
  a_\text{rechts} &= -71.03 \pm  6.37\\
  b_\text{rechts}&= 1141.29 \pm 76.73
\end{align*}
\noindent
Für die Halbwertsbreite dieser Messdaten wurde
\begin{equation*}
  t_{1/2}= \SI{14.7\pm 1.9}{\nano\second}
\end{equation*} 
bestimmt.


\subsection{Bestimmung der mittleren Lebensdauer der Myonen}
Zur Bestimmung der mittleren Lebensdauer der Myonen wurden die in \autoref{fig:myo} dargestellten Messwerte entsprechend einem Zerfallsgesetz einer Funktion der Form 
\begin{equation}
  N(t)=N_0\cdot e^{-\lambda t} \label{eqn:exp}
\end{equation}
angepasst.
Die gesamte Dauer der Messung betrug $t_\text{ges}=\SI{90920}{\second}$.
Es wurden sowohl am Anfang, als auch am Ende der Messdaten einige der Daten mit einem Messwert von N=0 für die Impulse entfernt.
Die Anzahl der gemessenen Impulse abhängig von der Zeit für diese Messung sind in \autoref{fig:myo} dargestellt.
\begin{figure}[h!]
    \centering
    \includegraphics[width=0.7\textwidth]{Abbildungen/myons.pdf}
    \caption{Anzahl der gemessenen Impulse in Abhängigkeit der Zeit mit entsprechender Ausgleichsfunktion der Form von \autoref{eqn:exp} für die Bestimmung der mittleren Lebensdauer der Myonen.}
    \label{fig:myo}
  \end{figure}
\noindent
Die Ausgleichsrechnung entsprechend der Funktion aus \autoref{eqn:exp} ergibt für die Parameter
\begin{align*}
  N_0    &=\num{37.531 \pm 1.321} \\
 \lambda &=\SI{0.506 \pm 0.024}{\per\second}.
\end{align*}
Aus $\lambda$ kann wie bereits in \autoref{sec:tau} beschrieben die mittlere Lebensdauer berechnet werden.
Für die mittlere Lebensdauer der Myonen ergibt sich so
\begin{equation*}
  \tau_{\mu}=\SI{1.977 \pm 0.092}{\micro\second}.
\end{equation*}