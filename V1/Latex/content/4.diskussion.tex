\section{Diskussion}
Zuerst wurde die \textbf{Kalibrierung des Multikanalanalysators} durchgeführt. Hier haben sich sehr geringe Unsicherheiten von nahezu 0 ergeben. Diese Kalibrierung hat also gut funktioniert und es sind keine systematischen Fehler durch die Umrechnung der Kanalnummern in Zeiten zu erwarten.
\newline
\noindent
Danach wurde die \textbf{Kalibrierung der Verzögerungsleitung} durchgeführt. Hier haben sich bei den Ausgleichsrechnungen für die Ausgleichsgeraden und das Plateau größere Fehlerwerte als bei der Kalibrierung des Multikanalanalysators ergeben. Daher ist hier von etwas größeren Ungenauigkeiten als bei der vorherigen Kalibrierung auszugehen. Grundsätzlich hat aber auch die Kalibrierung der Verzögerungsleitung ohne Probleme funktioniert.
Außerdem wurde für die Messwerte eine Halbwertsbreite von
\begin{equation*}
  t_{1/2}= \SI{14.7\pm 1.9}{\nano\second}
\end{equation*}  
bestimmt. Die ermittelte Halbwertsbreite liegt in einem sinnvollen Bereich für eine Pulsbreite von $\SI{10}{\nano\second}$.
\newline
\noindent
Anschließend wurde in \autoref{sec:aus} experimentell ein Wert für die \textbf{mittlere Lebensdauer der Myonen} bestimmt. Der bestimmte Wert für die Lebensdauer der Myonen ist 
\begin{equation*}
  \tau_{\mu}=\SI{1.977\pm0.092}{\micro\second}.
\end{equation*}
Der Literaturwert für die Lebensdauer von Myonen beträgt
\begin{equation*}
    \tau_{\mu,\text{Lit}}=\SI{2.197}{\micro\second} \quad\text{\cite{griff}}.    
\end{equation*}
Die prozentuale Abweichung der experimentell bestimmten mittleren Lebensdauer $\tau_{\mu}$ zu dem Literaturwert beträgt also in etwa 
\begin{equation*}
    p=\frac{|\tau_{\mu}-\tau_{\mu,\text{Lit}}|}{\tau_{\mu,\text{Lit}}}\approx \SI{10.01\pm 4}{\percent}.
\end{equation*}
Diese Abweichung ist nicht besonders groß. Es scheinen also keine großen Fehler bei der Durchführung des Versuchs aufgetreten zu sein. Die vorhandene Abweichung kann auf Ungenauigkeiten bei der Kalibrierung des Versuchsaufbaus oder auf kleinere in der Auswertung vernachlässigte Prozesse innerhalb des Szintillators zurückzuführen sein.