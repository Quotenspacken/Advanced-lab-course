\section{Theorie}
\subsection{Kosmische Myonen}
Myonen sind Elementartteilchen, die zur Familie der Leptonen gehören.
Sie besitzen eine negative elektrische Ladung von \(-1e\) und eine Ruheenergie von etwa \(105{,}7\,\mathrm{MeV}/c^2\), was ungefähr dem 207-fachen der Elektronenmasse entspricht.
Neben Teilchenbeschleunigern ist die kosmische Strahlung eine bedeutende Quelle für Myonen.
Hier entstehen aus hochenergetischen Primärteilchen, 
die in die Erdatmosphäre eindringen, 
durch Kollisionen mit Atomkernen sekundäre Teilchen wie Pionen.
Pionen zerfallen in den meisten fällen in Myonen und Myon-Neutrinos gemäß der Reaktionen:
\begin{align*}
    \pi^+ &\rightarrow \mu^+ + \nu_\mu \quad \\
    \pi^- &\rightarrow \mu^- + \bar{\nu}_\mu. 
\end{align*}
Anders als Elektronen sind Myonen instabil und zerfallen nach einer durchschnittlichen Lebensdauer von etwa \(2{,}2\,\mu\mathrm{s}\) \cite{griff}.
Der hüfigste Zerfallskanal ist der in ein Elektron, ein Elektron-Antineutrino und ein Myon-Neutrino:
\begin{equation*}
    \mu^- \rightarrow e^- + \bar{\nu}_e + \nu_\mu.      
\end{equation*}



\section{Definition der Lebensdauer}\label{sec:tau}
Die Lebensdauer \(\tau\) eines instabilen Teilchens ist definiert als der Mittelwert der Zeitintervalle zwischen der Entstehung und dem Zerfall des Teilchens.
Da jedes Teilchen eine individuelle Lebensdauer hat,
ist dieser Prozess statistisch Verteilt.
Diese Verteilung lässt sich durch die Zerfallskonstatnte \(\lambda\) charakterisieren, 
die den Zerfall pro Zeiteinheit beschreibt.
Im infinitesimalen Zeitintervall \(\mathrm{d}t\) ist die Wahrscheinlichkeit \(\mathrm{d}W\), dass ein Teilchen zerfällt, proportional zur Zerfallskonstanten und dem Beobachtungszeitintervall \(\mathrm{d}t\):
\begin{equation*}
    \mathrm{d}W = \lambda \, \mathrm{d}t.           
\end{equation*}
Da nun Teilchen unabhängig voneinander zerfallen und die Zerfallswahrscheinlichkeit nicht vom Alter des Teilchens abhängt, 
ergibt sich für die Zahl
\(N(t)\) der noch nicht zerfallenen Teilchen zum Zeitpunkt \(t\):
\begin{equation}
    dN = -N(t) \, \mathrm{d}W = -\lambda N(t) \, \mathrm{d}t.
    \label{eq:dN}
\end{equation}
Wenn N eine große Zahl ist,
dann lässt sich Gleichung \eqref{eq:dN} näherungsweise Integrieren.
Bildet man zussätzlich daraus die Verteilungsfunktion der Lebensdauer $t$,
so ergibt sie sich zu:
\begin{equation}
    \frac{dN(t)}{N_0} = -\lambda e^{-\lambda t} \, \mathrm{d}t,
    \label{eq:verteilung}
\end{equation}
der Exponentielle Verteilung mit $N_0$ als der Gesamtzahl der betrachteten Teilchen.
Um nun die charakteristische Lebensdauer \(\tau\) zu bestimmen,
wird der Mittelwert aller möglichen Lebensdauern berechnet:
\begin{equation}
    \tau = \int_0^\infty \lambda t e^{-\lambda t} \, \mathrm{d}t = \frac{1}{\lambda}.
    \label{eq:lebensdauer}
\end{equation}  

\subsection{Technik für Teilchenphysik}
\subsection{Photomultiplier}
Photomultiplier (PMTs) sind hochempfindliche Detektoren,
die einzelne Photonen in elektrische Signale umwandeln und verstärken können.       
Ein PMT besteht aus einer Photokathode,
die Photonen absorbiert und Elektronen emittiert.
Dabei tragend ist der sogenannte Photoeffekt.
Die emittierten Elektronen werden dann durch eine Reihe von Dynoden beschleunigt,       
die jeweils eine höhere Spannung haben als die vorherige.
Wenn ein Elektron auf eine Dynode trifft,
schlägt es mehrere Elektronen heraus,
dadurch wird die anfangs geringe menge an Elektronen vervielfacht \cite{Leo}.

\subsubsection{Szintillationsdetektoren}
Szintillationsdetektoren sind Geräte, die ionisierende Strahlung durch die Emission von Lichtblitzen nachweisen.
Sie bestehen aus einem Szintillationsmaterial, 
das bei der Wechselwirkung mit geladenen Teilchen Licht emittiert.
Dieses Licht kann dann von einem Photomultiplier (PMT) detektiert werden, 
der die Lichtsignale in elektrische Signale umwandelt.

Man unterscheidet zwischen organischen und anorganischen Szintillatoren,
die für unterschiedliche Anwendungen optimiert sind.
Organische Szintillatoren, wie z.B. Kunststoff- oder Flüssigszintillatoren,
reagieren schnell auf ionisierende Strahlung und eigen sich daher gut für Experimente,
die eine hohe Zeitauflösung erfordern.
Anorganische Szintillatoren, wie z.B. Natriumiodid (NaI) oder Bismutgermanat (BGO),
besitzen eine hohe Energieauflösung und sind besonders geeignet für die Untersuchung von hochenergetischen Teilchen\cite{Leo}.

\subsection{Diskriminatoren}
Diskriminatoren sind elektronische Bauelemente,
die Signale nach ihrer Amplitude filtern.
Sie geben nur dann ein Ausgangssignal ab,
wenn das Eingangssignal einen bestimmten Schwellenwert überschreitet.
Dadurch kann zum Beispiel das Rauschen unterdrückt werden.

\subsection{Multikanal-Analysatoren}
Multikanal-Analysatoren (MCA) sind Geräte,
die Signale in verschiedene Kanäle aufteilen und analysieren können.    
Sie werden häufig in der Teilchenphysik und Kernphysik eingesetzt,
um die Energieverteilung von Teilchen zu messen.
Ein MCA besteht aus einem Analog-Digital-Wandler (ADC),
der das analoge Eingangssignal in digitale Werte umwandelt.
Diese digitalen Werte werden dann in verschiedene Kanäle sortiert,
die jeweils eine bestimmte Energie oder Amplitude repräsentieren \cite{Leo}.