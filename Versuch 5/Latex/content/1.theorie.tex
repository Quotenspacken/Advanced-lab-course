\section{Objective}
\label{sec:Zielsetzung}

The objective of this experiment is to determine the energy and activity of several known gamma radiation sources and one unknown source using a germanium detector. For this purpose, an energy calibration of the germanium detector is first carried out, followed by the determination of the full-energy detection efficiency.

\section{Theory}
\label{sec:Theorie}


\subsection{Interaction of Photons with Matter}
\label{sec:process}

\noindent
Photons can interact with matter through various interaction processes, and the nature of these interactions depends significantly on the photon energy. The main processes that occur are the photoelectric effect, the Compton scattering, and the pair production. These effects will be described in more detail below.

\subsubsection*{Photoelectric Effect}
The photoelectric effect dominates at lower energies. In this process, the photon interacts with a valence electron of an atom. If the photons energy is greater than the binding energy of the valence electron, the electron leaves the atom and the photon is absorbed. The resulting vacancy is filled by electrons from higher energy levels.  
The energy released in the photoelectric effect is completely absorbed and usually remains entirely within the detector.

\subsubsection*{Compton Effect}

In Compton scattering, a photon transfers part of its energy to a free electron and changes its direction by an angle $\theta$. Thus, Compton scattering essentially describes elastic scattering of a photon by a free electron.  
The amount of transferred energy depends on the scattering angle $\theta$. The maximum transfer occurs at an angle of $\theta = \SI{180}{\degree}$ and is referred to as the Compton edge.
\noindent
The cross section for Compton scattering is given by the Klein–Nishina equation:
\begin{equation}
    \sigma_\text{Compton}=\frac{\pi\alpha^2}{m^2}\frac{1}{\hat{E}^3}\left(\frac{2\hat{E}(2+\hat{E}(1+\hat{E})(8+\hat{E}))}{(1+2\hat{E})^2}+((\hat{E}-2)\hat{E}-2)\log(1+2\hat{E})\right)
\end{equation}


\subsubsection*{Pair Production}
Pair production is the dominant effect at high energies. The cross section depends significantly on $\alpha$ and on the atomic number Z:

\begin{equation}
    \sigma_\text{Pair}\propto \alpha Z^2
\end{equation}
\newline
\noindent
In \autoref{fig:extinction}, the energy dependence of these three discussed processes is shown graphically. As described, it can be seen that at low energies, photoelectric absorption dominates, at high energies pair production dominates and in between Compton scattering is dominant.

\begin{figure}[h]
    \centering
    \includegraphics[width=\textwidth]{plots/strahlung.png}
    \caption{Energy dependence of different types of interactions with matter \cite{LNT}.}
    \label{fig:extinction}
\end{figure}
\noindent
From \autoref{fig:extinction}, it can be seen that for the energies relevant to this experiment, primarily photoelectric absorption and Compton scattering are significant.

\subsection{Spectrum of a Monochromatic \texorpdfstring{$\gamma$}{gamma} Source}
In a typical gamma spectrum, one can identify a photopeak (also called full-energy peak), a Compton edge, and a backscatter peak. An example gamma spectrum from a $^{137}\text{Cs}$ source is shown in \autoref{fig:spectrum}.

\begin{figure}[h]
    \centering
    \includegraphics[width=\textwidth]{plots/spektrum.png}
    \caption{Example gamma spectrum from a $^{137}\text{Cs}$ source showing 
    the full-energy peak, the Compton edge(Comptonkante), and the backscatter peak(Rückstreupeak) \cite{spectrum}.}
    \label{fig:spectrum}
\end {figure}
\noindent
At the photopeak, all photon energy is deposited in the detector. The energies with highest emission probability for elements investigated in this experiment are given in \cite{lara}:
\begin{itemize}
\item $^{152}\text {Eu}$ : $\SI {121.78}{\kilo\eV}$ % http://www.lnhb.fr/Laraweb/
\item $^{137}\text {Cs}$ : $\SI {661.66}{\keV}$ % http://www.lnhb.fr/Laraweb/
\item $^{133}\text {Ba}$ : $\SI {356.01}{\keV}$ % http://www.lnhb.fr/Laraweb/
\end{itemize}
The continuous part of the spectrum originates from the Compton effect with maximum energy transfer occurring at the Compton edge.
\newpage