\section{Experimental Setup and Procedure}
\label{sec:Au}

\subsection{Experimental Setup}
\label{sec:Aufbau}

In this experiment, a coaxial, cylindrical germanium detector with a diameter of $\SI{45}{\milli\meter}$ and a height of $\SI{39}{\milli\meter}$ is used. The schematic structure of the detector is shown in Figure \ref{fig:aufbau}.  
\begin{figure}[h]
\centering
\includegraphics[width=\textwidth]{Plots/Ge_detector.png} %[width=0.7\textwidth]
\caption{Cross section of the germanium detector used. \cite{V18}}
\label{fig:aufbau}
\end{figure}
\noindent
The surface of the detector is n-doped by lithium atoms diffused into the surface. Inside the detector crystal, there is a coaxial hole whose surface is coated with gold. This metal–semiconductor contact corresponds to a strong p-doping.  
Additionally, the entire detector crystal is surrounded by an insulating aluminum housing. As a result, there exists a lower detection limit for detectable gamma energies. Therefore, only energies greater than $\SI{150}{\kilo\electronvolt}$ should be used for determining the full-energy detection efficiency.

\subsection{Experimental Procedure}
\label{sec:Durchführung}
During the execution of this experiment, various radioactive sources are examined using a germanium detector as shown in \autoref{fig:aufbau}. The gamma spectra of four different samples are measured.  
\newline
\noindent
In the first part of the task, an energy calibration of the detector is carried out and the full-energy detection efficiency is determined. For this purpose, a $^{152}\text{Eu}$ source with precisely known activity at the time of production is investigated.  
\newline
\noindent
After that, a $^{137}\text{Cs}$ sample is examined. For this sample, an entire gamma spectrum is recorded and subsequently analyzed in more detail.  
\newline
\noindent
Next, the spectrum of a $^{133}\text{Ba}$ sample is recorded with the detector. From these measured data, the activity of the barium sample is determined.  
\newline
\noindent
Finally, the spectrum of an unknown source is measured. Based on the measurement data from the gamma spectrum of this unknown source, the active components of the emitter are identified.

\newpage