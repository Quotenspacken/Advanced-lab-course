\section{Analysis}
\subsection{Energy Calibration with Europium}
To calibrate the detector, 
the spectrum of a $^{152}$Eu source is recorded. 
The measurement is performed over a duration of 120 minutes, 
yielding the spectrum shown in \autoref{fig:eu_spectrum}. 
\begin{figure}[h!]
    \centering
    \includegraphics[width=0.7\textwidth]{../build/Europium_Spektrum.pdf}
    \caption{Spectrum of the $^{152}$Eu source. The highest peaks are labeled with a red cross. The background spectrum has already been subtracted.}
    \label{fig:eu_spectrum}
\end{figure}
In addition,
a background measurement is conducted over 24 hours. 
To obtain the net counts, 
the background spectrum is first scaled to the same measurement time as the $^{152}$Eu spectrum and subsequently subtracted. 
The recorded background spectrum is shown in \autoref{fig:background_spectrum}.
\begin{figure}[h!]      
    \centering
    \includegraphics[width=0.7\textwidth]{../build/Hintergrund_Spektrum.pdf}
    \caption{Background spectrum measured over 24 hours.}
    \label{fig:background_spectrum} 
\end{figure}
For the determination of the energy calibration, 
the peaks in the $^{152}$Eu spectrum are identified. 
The corresponding gamma energies are known from literature \cite{laraweb}.
Peak identification is performed using the \textit{find peaks} function provided by the scipy library \cite{scipy}. 
To suppress statistical fluctuations and electronic noise, 
only peaks exceeding $3\%$ of the maximum peak height are considered. 
The identified peaks are indicated in \autoref{fig:eu_spectrum} by red crosses.

The energy calibration is obtained via a linear fit of the form
\begin{equation}
	E(K) = \alpha \cdot K + \beta
	\label{eqn:linear}
\end{equation}
where $K$ denotes the channel number and $E$ the corresponding energy. 
The resulting fit parameters are
\begin{align*}
	\alpha & =  0.1191 \pm 0.0000      \\
	\beta  & =  -0.8742 \pm 0.0004.
\end{align*}
The calibrated spectrum obtained using this fit is shown in \autoref{fig:eu_calibration}.
\begin{figure}[h!]
    \centering
    \includegraphics[width=0.7\textwidth]{../build/Europium_Spektrum_konfiguriert.pdf}
    \caption{Spectrum of the $^{152}$Eu source with the energy calibration applied.The energies of the characteristic peaks from the literature are plotted on the x-axis \cite{laraweb}.}
    \label{fig:eu_calibration}
\end{figure}
The relation~\eqref{eqn:linear} with the determined parameters is used throughout this analysis for energy measurements.



\subsection{Full energy detection probability}
The full energy detection probability $Q$  quantifies the probability that a photon emitted by the source is not only geometrically incident on the detector but also deposits its complete energy within the sensitive detector volume. 
The full energy detection probability is defined as
\begin{equation}
	Q = \frac{4 \pi}{\Omega}\frac{N}{APt},
	\label{eqn:fullenergy}
\end{equation}
where $N$ denotes the number of photons detected in the respective full energy peak. 
The quantity $A$ represents the activity of the source at the time of measurement, 
$P$ is the emission probability of the corresponding gamma transition, 
and $t$ is the measurement time. 
The factor $\Omega$ describes the solid angle under which the detector is seen from the position of the source. 

For the germanium detector used in this measurement, 
the solid angle fraction $\Omega / 4\pi$ is given by
\begin{equation}
	\frac{\Omega}{4 \pi} = \frac{1}{2} \left( 1 - \frac{a}{\sqrt{a^{2} + r^{2}}} \right) =  0.01522 \pm 0.00034,
	\label{eqn:solidangle}
\end{equation}
where $r = \SI{2.25 \pm 0.1}{\centi\meter}$ denotes the detector radius and $a = \SI{8.01 \pm 0.1}{\centi\meter}$ the distance between the source and the detector surface. 

The activity was calculated based on the known value that on 01.10.2000 the activity of the Europium sample was $A = \SI{4130(60)}{\becquerel}$. 
Using the half-life $t_{\sfrac{1}{2}} = \SI{4943(5)}{\day}$, 
the activity at the day of measurement is obtained via the exponential decay law
\begin{align*}
	A = A_0 \exp\left(-\frac{\ln{2} \cdot\Delta t}{t_{\sfrac{1}{2}}}\right)
	= 1124.17 \pm 24 \text{Bq}.
\end{align*}
Here, $\Delta t$ denotes the time elapsed between the reference date and the measurement date. 

The number of detected photons $N$ in each full energy peak was extracted by fitting a Gaussian function of the form
\begin{align*}
	f\left(x\right) = h\cdot \exp{\frac{(x-\mu)^2}{2\sigma^2}} + a
\end{align*}
to the corresponding spectral region. 
The fits were performed for each relevant energy peak using the Python package $\textit{curve\_fit}$. 
The integral of the Gaussian component yields the peak content $N$, which is subsequently used in \autoref{eqn:fullenergy}.

The measured full energy detection probability as a function of energy is shown in \autoref{fig:full_energy}. 
To parameterize the energy dependence, a power function is fitted to the data. 
The fitted curve is indicated by the red line and provides a continuous description of the detector efficiency over the considered energy range.
\begin{figure}[h!]
    \centering
    \includegraphics[width=0.7\textwidth]{../build/Effizienz.pdf}
    \caption{Full energy detection probability $Q$ as a function of energy. The fit of the power function is shown as a red line.}
    \label{fig:full_energy}     
\end{figure}



\subsection{Gamma spectrum of Ceasium}
A $^{137}\text{Cs}$ probe was measured over a Time period of $t=\SI{14400}{\second}$.
The resulting spectrum is shown in \autoref{fig:cs_spectrum}.
With red crosses, from left to right, the x-ray peak, the backscattered peak, the Compton edge, and the full energy peak are marked.
\begin{figure}[h!]
    \centering
    \includegraphics[width=0.7\textwidth]{../build/Cs_Spektrum.pdf}
    \caption{Spectrum of the $^{137}\text{Cs}$ source. The background spectrum has already been subtracted.}
    \label{fig:cs_spectrum} 
\end{figure}
A scaled normal distribution is fitted to the full energy peaks to determine the yield. 
The results is displayed in Figure~\ref{fig:csfit}.
\begin{figure}[h!]
    \centering
    \includegraphics[width=0.7\textwidth]{../build/Cs_full_peak.pdf}
    \caption{Fit of a scaled normal distribution to the full energy peak of the $^{137}\text{Cs}$ spectrum.}
    \label{fig:csfit}               
\end{figure}
These fits include a further background estimation parameter $b$\,; the resulting distribution is given by
\begin{equation}
	g(x) = s \cdot \frac{1}{\sqrt{2 \pi} \sigma} \exp{\left( -\frac{1}{2} \frac{(x - \mu)^{2}}{\sigma} \right)} + b.
	\label{eqn:fitplus}
\end{equation}
The fit parameters including the number of events of each peak is given in Table~\ref{tab:csfitparams}; the yield of each peak is given by the scale factor $s$.
\begin{table}[ht]
	\centering
	\caption{Fit parameters of the photo peaks of the $^{137}\text{Cs}$ spectrum with more than $\SI{150}{\kilo\electronvolt}$.}
	\label{tab:csfitparams}
	\sisetup{table-format=3.5}
	\begin{tabular}{S S[table-format=4.3] S[table-format=2.0] S[table-format=4.3] S[table-format=2.3]}
		\toprule {$E/\si{\kilo\electronvolt}$} & {$s$}                 & {$b$}              & {$\mu$}              & {$\sigma$}         \\
		\midrule
		{$661.3941 \pm 0.0004$} & {$9283 \pm 98$} & {$15.00 \pm 0.17$} & {$3195.88 \pm 0.05$} & {$4.49 \pm 0.04$}  \\
		\bottomrule
	\end{tabular}
\end{table}
The full width at half maximum and full width at tenth maximum of the photo peak of the $^{137}\text{Cs}$ spectrum are estimated. 
To do this, the fitted normal distribution is utilized. 
This is also displayed in \autoref{fig:csfit}.

The full width at half maximum and the full width at tenth maximum amount to
\begin{align}
	\text{FWHM} & = \SI{2.2 \pm 0.020}{\kilo\electronvolt}  \\
	\text{FWTM} & = \SI{4.0 \pm 0.030}{\kilo\electronvolt}.
\end{align}
The ratio of the two is given by
\begin{equation}
	\frac{\text{FWHM}}{\text{FWTM}} = \SI{0.55 \pm 0.0004}{\kilo\electronvolt}.
\end{equation}
The energy of the photoelectric peak is determined from the fit to
\begin{equation}
	E_{\gamma} = \SI{661.3941 \pm 0.0004}{\kilo\electronvolt}.
	\label{eqn:egamma}
\end{equation}
The energy of the compton edge can be calculated by the Compton scattering formula
\begin{equation}
	E_{\gamma}^{\prime} \vert_{\theta=\SI{180}{\degree}} = \frac{E_{\gamma}}{1 + \frac{m_{e} c^{2}}{2 E_{\gamma}}}.
	\label{eqn:emax}
\end{equation}
This amounts to
\begin{equation}
	E_{\gamma}^{\prime}\vert_{\theta=\SI{180}{\degree}} = \SI{477.271 \pm 0.0004}{\kilo\electronvolt}.
\end{equation}

The compton edge is estimated to be at $E_{\gamma}^{\prime}\vert_{\theta=\SI{180}{\degree}} = \SI{464.742}{\kilo\electronvolt}$. 
The energy region below the compton edge should be dominated by the compton effect and include minimal amounts of background. 
The backscatter peak is estimated to be at $E_{\gamma} = \SI{190.96 \pm 0.0004}{\kilo\electronvolt}$. 
To estimate the event count in the compton continuum, 
a linear fit is carried out in the region between the backscatter peak and the compton edge.
The fit is displayed in Figure~\ref{fig:comptonfit}.
\begin{figure}[h!]
	\centering
	\includegraphics[scale=0.7]{../build/Caesium-Compton-Fit.pdf}
	\caption{Linear Fit of the compton continuum.}
	\label{fig:comptonfit}
\end{figure}
\noindent
The function utilized here is given in Equation
\begin{equation}
	f(x) = \alpha \cdot x + \beta .
	\label{eqn:comptonfit}
\end{equation}
The parameters are estimated to the values
\begin{align}
	\alpha & = \num{0.008 \pm 0.000}    \\
	\beta  & = \num{1.0 \pm 0.8}.
\end{align}
This yield of the compton continuum as estimated by this fit amounts to
\begin{equation}
	N = \num{15498.84830 \pm 0.00000}.
\end{equation}



\subsection{Gamma spectrum of Barium}
The spectrum of a $^{133}\text{Ba}$ source is measured over a time period of $t = \SI{7200}{\second}$.
The resulting spectrum is shown in \autoref{fig:ba_spectrum}.
\begin{figure}[h!]
    \centering
    \includegraphics[width=0.7\textwidth]{../build/Ba_Spektrum.pdf}
    \caption{Spectrum of the $^{133}\text{Ba}$ source. The background spectrum has already been subtracted.}
    \label{fig:ba_spectrum}
\end{figure}
Again, 
the background measurement is subtracted. 
The algorithm \textit{find\_peaks} from the python library \texttt{scipy} \cite{scipy} is utilized to determine the exact positions of the peaks.

To estimate the activity of the $^{133}\text{Ba}$ source, Equation~\eqref{eqn:fullenergy} is used. 
The full energy Equation~\eqref{eqn:fullenergy} is rearranged to be able to compute the probe activity:
\begin{equation}
	Q = \frac{4 \pi}{\Omega}\frac{N}{APt} \leftrightarrow A = \frac{4 \pi}{\Omega} \frac{N}{QPt}.
	\label{eqn:full-energy-detection-efficiency}
\end{equation}
\begin{table}
	\centering
	\caption{Calculated activity of the emission lines under consideration with corresponding detector efficiencies.}
	\label{tab:aktivitaet_ba}
	\begin{tabular}{
		S[table-format=2.1]
		S[table-format=1.3] @{${}\pm{}$} S[table-format=1.3]
		S[table-format=5.2] @{${}\pm{}$} S[table-format=2.2]
		S[table-format=4.0] @{${}\pm{}$} S[table-format=2.0]
		}
	\toprule
		{$\text{Intensity}\/\%$} &
		\multicolumn{2}{c}{Q} &
		\multicolumn{2}{c}{$E_i$\;/\;\si{\kilo\electronvolt}} &
		\multicolumn{2}{c}{$A_i$\;/\;\si{\becquerel}} \\
	\midrule
		 34.1 &  0.000 &  0.000 &    72.96  &  0.01 &    0 &  0 \\
		 18.3 &  0.280 &  0.005 &    296.16 &  0.00 &  1096&  25 \\
		 62.1 &  0.228 &  0.002 &    349.66 &  0.00 &  964 &  21 \\
		 8.9  &  0.163 &  0.007 &    377.58 &  0.00 &  902 &  21 \\
	\bottomrule
	\end{tabular}
\end{table}

After assigning the peaks, 
a Gaussian fit was used to obtain the peak heights of the individual energies. 
The formula \eqref{eqn:full-energy-detection-efficiency}
was then used to determine the activity of the sample on the day of measurement. 
From the Gaussian parameters and the values calculated from them in Table \ref{tab:aktivitaet_ba}, 
the mean activity was determined by forming a mean value of
\begin{align*}
  A = \SI{987(23)} {\becquerel}.
\end{align*}



\subsection{Spectrum of an unknown source}
A measurement of an unknown source is conducted,
over a time period of $t = \SI{7200}{\second}$.
The measured spectrum is displayed in Figure~\ref{fig:urcomp}.
Through comparing the peaks of the spectra, with the peaks of the Background spectrum in \autoref{fig:background_spectrum}, 
the ones which likely originate from the  probe can be identified. 
These peaks are shown in Figure~\ref{fig:urcomp}.
\begin{figure}[h!]
	\centering
	\begin{adjustbox}{width=0.7\textwidth, center}
		\includegraphics[scale=0.3]{../build/Uran-Uran-Peaks.pdf}
	\end{adjustbox}
	\caption{Spectrum of the uranium measurement, with background peaks removed.}
	\label{fig:urcomp}
\end{figure}
\noindent
A normal distribution is fitted to the remaining peaks to estimate the yield.
The emission lines are matched to possible nuclides from the literature~\cite{laraweb}. 
A possible matching is given in Table~\ref{tab:urmatching}.
\begin{table}[h!]
	\centering
	\caption{Matching of peaks of the unknown uranium sources´ spectrum.}
	\label{tab:urmatching}
	\sisetup{table-format=4.0}
	\begin{adjustbox}{width=1.3\textwidth, center}
		\begin{tabular}{S[table-format=2.0] S[table-format=4.4] S[table-format=3.3] S[table-format=1.5] S[table-format=3.2] S[table-format=1.4] S[table-format=3.2] S[table-format=1.4] S[table-format=3.2] S[table-format=1.4]}
			\toprule {} & {}                                      & \multicolumn{2}{c}{$^{233}\text{U}$} & \multicolumn{2}{c}{$^{226}\text{Ra}$} & \multicolumn{2}{c}{$^{214}\text{Pb}$} & \multicolumn{2}{c}{$^{214}\text{Bi}$}                                                                                                                   \\
			\hline
			{Index}     & {$\text{Peak}/\si{\kilo\electronvolt}$} & {Peak$/\si{\kilo\electronvolt}$}     & {P$/\si{\percent}$}                   & {Peak$/\si{\kilo\electronvolt}$}      & {P$/\si{\percent}$}                   & {Peak$/\si{\kilo\electronvolt}$} & {P$/\si{\percent}$} & {Peak$/\si{\kilo\electronvolt}$} & {P$/\si{\percent}$} \\
			\midrule
			0           & {$63.1025 \pm 0.0004$}                    & {$62.950 \pm 0.017$}                   & {0.49}                                & {$62.5 \pm 1.0$}                        & {0.0116}                              & {$62.5 \pm 1.0$}                   & {0.0116}            & {$62.5 \pm 1.0$}                   & {0.0116}            \\
			1           & {$87.1670 \pm 0.0004$}                    & {$87.385 \pm 0.016$}                   & {0.271}                               & {-}                                   & {-}                                   & {-}                              & {-}                 & {-}                              & {-}                 \\
			2           & {$111.0239 \pm 0.0004$}                   & {$111.517 \pm 0.018$}                  & {0.313}                               & {-}                                   & {-}                                   & {-}                              & {-}                 & {-}                              & {-}                 \\
			3           & {$143.8014 \pm 0.0004$}                   & {$144.627 \pm 0.042$}                  & {0.00046}                             & {-}                                   & {-}                                   & {-}                              & {-}                 & {-}                              & {-}                 \\
			4           & {$153.9665 \pm 0.0004$}                   & {$153.955 \pm 0.018$}                  & {0.205}                               & {-}                                   & {-}                                   & {-}                              & {-}                 & {-}                              & {-}                 \\
			5           & {$235.7026 \pm 0.0004$}                   & {-}                                  & {-}                                   & {-}                                   & {-}                                   & {-}                              & {-}                 & {-}                              & {-}                 \\
			6           & {$241.7187 \pm 0.0004$}                   & {$240.663 \pm 0.021$}                  & {0.0117}                              & {-}                                   & {-}                                   & {-}                              & {-}                 & {-}                              & {-}                 \\
			7           & {$258.9372 \pm 0.0004$}                   & {-}                                  & {-}                                   & {$258.87 \pm 0.03$}                     & {0.5318}                              & {$258.87 \pm 0.03$}                & {0.5318}            & {-}                              & {-}                 \\
			8           & {$268.6875 \pm 0.0004$}                   & {-}                                  & {-}                                   & {-}                                   & {-}                                   & {-}                              & {-}                 & {$268.8 \pm 0.2$}                  & {0.0161}            \\
			9           & {$767.4021 \pm 0.0004$}                   & {-}                                  & {-}                                   & {-}                                   & {-}                                   & {-}                              & {-}                 & {$768.356 \pm 0.010$}              & {4.892}             \\
			10          & {$805.1583 \pm 0.0004$}                   & {-}                                  & {-}                                   & {-}                                   & {-}                                   & {-}                              & {-}                 & {$806.174 \pm 0.018$}             & {1.262}             \\
			11          & {$933.5711 \pm 0.0004$}                   & {-}                                  & {-}                                   & {-}                                   & {-}                                   & {-}                              & {-}                 & {$934.061 \pm 0.012$}              & {3.10}              \\
			12          & {$1237.6958 \pm 0.0004$}                  & {-}                                  & {-}                                   & {-}                                   & {-}                                   & {-}                              & {-}                 & {$1238.111 \pm 0.012$}             & {5.831}             \\
			13          & {$1377.3110 \pm 0.0004$}                  & {-}                                  & {-}                                   & {-}                                   & {-}                                   & {-}                              & {-}                 & {$1377.669 \pm 0.012$}             & {3.968}             \\
			\bottomrule
		\end{tabular}
	\end{adjustbox}
\end{table}
\noindent
The unknown source likely consists of $^{233}\text{U}$, $^{214}\text{Bi}$ and its
daughter nuclides $^{226}\text{Ra}$ and $^{214}\text{Pb}$, as these would inhibit
all measured emission lines which could be attributed to the source. For each nuclide
and peak the activity is calculated using the literature values given in Table~\ref{tab:urmatching}.
The resulting activitys are given in Table~\ref{tab:urresults}.
\begin{table}[h!]
	\centering
	\caption{Results of the activity calculation based on the aforementioned peaks in the measured spectrum, ordered by presumed nuclide.}
	\label{tab:urresults}
	\sisetup{table-format=1.0}
	\begin{adjustbox}{width=1.3\textwidth, center}
		\begin{tabular}{S[table-format=2.0] S[table-format=4.4] S[table-format=6.4] S[table-format=2.4] S S[table-format=4.4]}
			\toprule {Index} & {$\text{Peak}/\si{\kilo\electronvolt}$} & {$A(^{233}\text{U})/\si{\kilo\becquerel}$} & {$A(^{226}\text{Ra})/\si{\kilo\becquerel}$} & {$A(^{214}\text{Pb})/\si{\kilo\becquerel}$} & {$A(^{214}\text{Bi})/\si{\kilo\becquerel}$} \\
			\midrule
			0                & {$63.1025 \pm 0.0004$}                    & {$63 \pm 6$}                                 & {-}                                         & {-}                                         & {-}                                         \\
			1                & {$87.1670 \pm 0.0004$}                    & {$202 \pm 19$}                               & {-}                                         & {-}                                         & {-}                                         \\
			2                & {$111.0239 \pm 0.0004$}                   & {$162 \pm 25$}                               & {-}                                         & {-}                                         & {-}                                         \\
			3                & {$143.8014 \pm 0.0004$}                   & {$130882 \pm 13486$}                         & {-}                                         & {-}                                         & {-}                                         \\
			4                & {$153.9665 \pm 0.0004$}                   & {$51 \pm 5$}                                 & {-}                                         & {-}                                         & {-}                                         \\
			5                & {$235.7026 \pm 0.0004$}                   & {-}                                        & {-}                                         & {-}                                         & {-}                                         \\
			6                & {$241.7187 \pm 0.0004$}                   & {$11624 \pm 1190$}                           & {-}                                         & {-}                                         & {-}                                         \\
			7                & {$258.9372 \pm 0.0004$}                   & {-}                                        & {$73 \pm 13$}                                 & {-}                                         & {-}                                         \\
			8                & {$268.6875 \pm 0.0004$}                   & {-}                                        & {-}                                         & {-}                                         & {$1033 \pm 106$}                    \\
			9                & {$767.4021 \pm 0.0004$}                   & {-}                                        & {-}                                         & {-}                                         & {$19.3 \pm 2.2$}                        \\
			10               & {$805.1583 \pm 0.0004$}                   & {-}                                        & {-}                                         & {-}                                         & {$56 \pm 6$}                        \\
			11               & {$933.5711 \pm 0.0004$}                   & {-}                                        & {-}                                         & {-}                                         & {$7.4 \pm 0.9$}                         \\
			12               & {$1237.6958 \pm 0.0004$}                  & {-}                                        & {-}                                         & {-}                                         & {$8.68 \pm 1.06$}                         \\
			13               & {$1377.3110 \pm 0.0004$}                  & {-}                                        & {-}                                         & {-}                                         & {$17.6 \pm 2.2$}                        \\
			\bottomrule
		\end{tabular}
	\end{adjustbox}
\end{table}
\noindent