\section{Discussion}
The detector calibration yields consistent and reliable results, 
which is of central importance since all subsequent analyses rely directly on the accuracy of the energy determination. 
The small uncertainties of the fit parameters contribute significantly to minimizing error propagation in the following calculations.
The calculated activity of the europium source is of the expected order of magnitude and therefore supports the consistency of the calibration procedure. 
The determination of the full energy detection probability follows the anticipated energy dependence. 

A primary limitation of the analysis of the europium spectrum—as well as of all subsequent spectral evaluations—is the rather simplified background treatment. 
The results could likely be improved substantially by performing a dedicated background fit instead of relying on a rudimentary subtraction. 
In addition, 
the use of a more sophisticated peak model could enhance the accuracy of the extracted peak parameters. 

Apart from these potential improvements, 
the fits of the caesium peaks and the determination of the full width at half maximum and tenth width at half maximum appear physically plausible and internally consistent. 
The Compton edge, 
however, 
can only be estimated approximately. 
Increasing the measurement time would improve the statistical quality of the spectrum, 
enhance the contrast in the relevant region, 
and thus allow for a more precise determination. 
Likewise, 
a more detailed modeling of the Compton continuum and the influence of the backscatter peak would benefit from additional dedicated fits of both the backscatter contribution and the residual background.

The analysis of the Barium spectrum also provides satisfactory results, 
aside from the aforementioned limitations in the background estimation. 

The identification of the unknown source is likely affected most strongly by the non-ideal background treatment.
A more refined comparison between the measured spectrum and the background spectrum could improve the peak assignment. 
Overall, 
the proposed nuclide composition provides a reasonable explanation of the measured spectrum. 
Nevertheless, 
it cannot be excluded that alternative combinations with higher emission probabilities might describe the data equally well or even better, 
particularly in cases where some of the considered emission lines exhibit comparatively low branching ratios.