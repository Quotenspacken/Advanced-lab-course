\section{Durchführung}
Der Versuchsaufbau ist wie in Abbildung \ref{fig:aufbau} dargestellt aufgebaut.
\begin{figure}[h!]               
    \centering
    \includegraphics[width=0.7\textwidth]{Abbildungen/Aufbau.png}
    \caption{Schematischer Aufbau des Versuchs zur Messung der Myonenlebensdauer \cite{anleitung}.}
    \label{fig:aufbau}
\end{figure}
Oben befindet sich ein 50 Litter großer Tank,
der mit in Wasser gelöstem Szintillator gefüllt ist.ß09
An den Enden sind zwei PMTs angebracht.
Das Ausgegebene Signal der PMTs wird zuerst über die Diskriminatoren geleitet,
die das Rauschen unterdrücken. 
Anschließend werden die Signale in einem Koinzidenzgerät zusammengeführt.
Dieses gibt nur einen Output, +
wenn beide PMTs gleichzeitig ein Signal empfangen.
Dies diehnt ebenfalls der Rauschunterdrückung.
Denn PMTs können auch durch Umwelteinflüsse wie z.B. Wäreme Signale erzeugen.
Da dies jedoch für beide PMTs gleichzeitig sehr unwahrscheinlich ist sind signale,
welche durch beide PMTs gleichzeitig empfangen werden mit hoher Wahrscheinlichkeit echte Ereignisse.

Zuletzt werden die Signale in einem Zeitmessgerät (TAC) ausgewertet.
Dieses misst die Zeit zwischen dem Eintreffen des ersten Signals (Myon kommt in den Tank) und dem zweiten Signal (Myon zerfällt).
Die Ausgabe des TACs wird in einem Multikanalanalysator gespeichert.



\subsection{Kallibration des Multikanalanalysator}
Zur Kallibration des Multikanalanalysators wird ein Doppelpulsgenerator verwendet.
Dieser erzeugt zwei Signale mit einem einstellbaren Zeitabstand.
Der Zeitabstand wird in 0.5 ns Schritten von 0 bis 55 ns variiert.  %Ich weiß nicht ob das so stimmt. Kannst du bitte nachsehen?
Die jeweiligen Kanäle des Multikanalanalysators werden den Zeitabständen zugeordnet.   



\subsection{Kallibration der Verzögerungsleitung}
Um möglichst viele Signale zu erfassen,
müssen die Signale der beiden PMTs zeitlich aufeinander abgestimmt werden.
Denn wenn ein ionisierendes Teilchen den Tank nicht mittig durchquert,
existiert eine Zeitdifferenz zwischen den Signalen der beiden PMTs.
Um die Einstellung der Verzögerungsleitung zu optimieren,
werden systematisch verschiedene Verzögerungszeiten an den Delays nach den PMTs eingestellt.
Die Zählrate  wird für jede Einstellung 60 Sekunden lang gemessen.
Die Verzögerungszeit mit der höchsten Zählrate wird für die eigentliche Messung verwendet


\subsection{Messung der Myonenlebensdauer}
Die Messung der Myonenlebensdauer erfolgt über einen Zeitraum von 24 Stunden.
Die Daten werden im Multikanalanalysator gespeichert und anschließend ausgewertet.
