\section{Theory}

\subsection{Coherent light}
The coherence of light is a crucial factor for observing clear interference patterns when two light waves interfere with each other. These patterns typically occur when two waves with compatible phases interfere and form a constructive or a destructive interference. Coherent light means that there is a consistent phase relationship between the interfering waves.
A laser beam, as used in this experiment, shows a high temporal and spatial coherence. The level of coherence can be calculated by:
\begin{equation}
    \gamma_{1,2}(\vec{r}_1,t_1;\vec{r}_2,t_2)=
    \frac{\langle E(\vec{r}_1,t_1)E^*(\vec{r}_2,t_2) \rangle}{\sqrt{\langle |E(\vec{r}_1,t_1)|^2 \rangle\langle |E(\vec{r}_2,t_2)|^2 \rangle}}
    \label{eqn:degree}
\end{equation}
Here $E(\vec{r}_i,t_i)$ represents the amplitude of the electric field at position $\vec{r}_i$ 
and time $t_i$. A $\gamma_{1,2}$ value of \num{0} shows complete incoherence, while 
\num{1} indicates perfect coherence.

\subsection{Polarization of light}
\label{sec:k}
Electromagnetic waves are oscillations in the electromagnetic field with different oscillation directions which determine their polarization. Unpolarized light typically consists of many different polarization directions. \newline
One possibility to get polarized light out of an unpolarized beam is letting it pass through a polarization filter, which only allows one specific polarization to pass through. Another way to get polarized light is to use a polarizing beam splitter cube (PBSC). Such a cube can reflect the s-polarized part of the light and transmit the p-polarized part of the beam. \newline


The contrast $K$ of the interference pattern is defined as
\begin{equation}
  K = \frac{I_{\symup{max}}-I_{\symup{min}}}{I_{\symup{max}}+I_{\symup{min}}} \in [0,\, 1],
  \label{eqn:1}
\end{equation}
with 0 being the minimal contrast and 1 the maximum. In general the intensity $I$ can be written as the superposition of two waves:
\begin{equation}
    I\propto \langle |E_1\cos(\omega t)+E_2\cos(\omega t +\delta)|^2 \rangle
    \label{eqn:intensity}
\end{equation}
With $\omega$ being the wave frequency and $\delta$ the phase difference between both waves. So the $I_{\symup{max}$ and $I_{\symup{min}$ can be written as
\begin{align*}
    I_\text{max/min}\propto\frac{1}{2}(E_1^2+E_2^2)\pm E_1E_2.
\end{align*}
And the electric field depending on the polarization angle $\phi$ of the incoming beam is given by
\begin{align*}
    E_1&=\sqrt{E_1+E_2}\cos(\phi) \\
    E_2&=\sqrt{E_1+E_2}\sin(\phi).
\end{align*}
Putting all of this together in \eqref{eqn:intensity} leads to 
\begin{equation*}
    I_\text{max/min}\propto (1\pm 2\sin(\phi)\cos(\phi)).
\end{equation*}
With all of this the contrast $K$ can be written depending on the polarization angle in the following way:
\begin{equation}
    K=2|\sin(\phi)\cos(\phi)|
    \label{eqn:k2}
\end{equation}

\subsection{Refractive indices}
\label{sec:ind}
The refractive index $n$ of a material describes how the speed of light in a medium $v$ 
changes compared to its speed in a vacuum $c$. It can be calculated by

\begin{equation*}
    n=\frac{v}{c}.
\end{equation*}

A way to calculate the number of intensity maxima and minima, $M$, for two interfering lightwaves with a phase difference of $\Delta\Phi$ is 
\begin{equation}
    M=\frac{\Delta\Phi}{2\pi}.
    \label{eqn:m}
\end{equation}
This phase difference $\Delta\Phi$ can be calculated for glass ($D$ = Thickness of glass plates) by
\begin{equation}
    \Delta\Phi(\theta)=2\pi\frac{D}{\lambda_\text{vac}}\frac{n-1}{2n}\theta^2
    \label{eqn:delta1}
\end{equation}
So inserting \eqref{eqn:delta1} in \eqref{eqn:m} leads to 
\begin{equation}
    M = \frac{D}{\lambda_\text{vac}}\frac{n-1}{n}\cdot 2\theta_0\theta.
    \label{eqn:M_ana}
\end{equation}
And solving this equation for the refractive index $n$ yields
\begin{equation}
    n = \frac{2\theta_0\theta D}{2\theta_0\theta D - \lambda_\text{vac} M}.
    \label{eqn:n}
\end{equation}

\noindent
For gases the phase difference can be calculated by
\begin{equation*}
    \Delta\Phi=2\pi\frac{L}{\lambda_\text{vac}}(n-1).
\end{equation*}
Here $L$ is the length of the used gas chamber. Another way for the calculation of the refractive index of a gas depending on its pressure $p$ and temperature $T$ is the Lorentz-Lorenz law:
\begin{equation}
    \frac{n^2-1}{n^2+1}=\frac{Ap}{RT}
    \label{eqn:lll}
\end{equation}
Here $R$ represents the universal gas constant and $A$ represents the molar refractivity of the gas.