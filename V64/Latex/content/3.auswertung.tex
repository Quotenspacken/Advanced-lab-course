\section{Analysis}



\subsection{Determination of the Contrast in dependency of the polarisation angle}
As described in \autoref{sec:k} the interference contrast $K$ is determined from the minimum and maximum intensities of the interference pattern according to
\begin{equation*}
    K = \frac{I_{\max}-I_{\min}}{I_{\max}+I_{\min}}.
\end{equation*}
To investigate the dependence of the contrast on the polarization angle $\phi$, 
the minimum and maximum intensities were measured for a range of polarization angles between $0^\circ$ and $180^\circ$. 
For each angle, three independent measurements were performed in order to reduce statistical fluctuations.
The measured values are summarized in Table~\ref{tab:contrast}. 
\begin{table}
    \centering
    \begin{tabular}{S S S S S S S}
        \toprule
        {$\phi/\si{\degree}$} & {$I_\text{min1}/\si{\volt}$} & {$I_\text{max1}/\si{\volt}$} & {$I_\text{min2}/\si{\volt}$} & {$I_\text{max2}/\si{\volt}$}  & {$I_\text{min3}/\si{\volt}$} & {$I_\text{max3}/\si{\volt}$}\\
        \midrule
        0   & 1.63 & 1.52 & 1.81 & 1.50 & 1.75 & 1.49 \\
        10  & 1.56 & 1.01 & 1.49 & 0.99 & 1.51 & 0.91 \\
        20  & 1.35 & 0.48 & 1.36 & 0.47 & 1.39 & 0.46 \\
        30  & 1.20 & 0.23 & 1.18 & 0.22 & 1.29 & 0.20 \\
        40  & 1.26 & 0.09 & 1.27 & 0.11 & 1.23 & 0.99 \\
        50  & 1.48 & 0.04 & 1.48 & 0.05 & 1.49 & 0.05 \\
        60  & 1.73 & 0.13 & 1.78 & 0.09 & 1.79 & 0.99 \\
        70  & 2.10 & 0.35 & 2.13 & 0.31 & 1.93 & 0.32 \\
        80  & 2.23 & 0.81 & 2.07 & 0.76 & 2.23 & 0.86 \\
        90  & 2.26 & 1.62 & 2.14 & 1.53 & 2.11 & 1.51 \\
        100 & 2.29 & 2.14 & 2.90 & 1.38 & 2.70 & 1.45 \\
        110 & 3.87 & 0.85 & 3.85 & 0.87 & 3.91 & 0.93 \\
        120 & 4.72 & 0.40 & 4.54 & 0.40 & 4.94 & 0.40 \\
        130 & 5.49 & 0.17 & 5.32 & 0.17 & 5.50 & 0.18 \\
        140 & 5.30 & 0.27 & 5.23 & 0.21 & 5.06 & 0.22 \\
        150 & 4.91 & 0.59 & 4.76 & 0.50 & 4.61 & 0.52 \\
        160 & 3.98 & 1.08 & 3.88 & 0.96 & 4.10 & 1.01 \\
        170 & 2.82 & 1.36 & 2.78 & 1.37 & 2.98 & 1.41 \\
        180 & 1.75 & 1.56 & 1.70 & 1.49 & 1.81 & 1.56 \\
        \bottomrule
    \end{tabular}
    \caption{Measured minimal and maximal intensity for different polarisation angles $\phi$.
             The measurement results from three different measurements are shown.}
    \label{tab:contrast}
\end{table}
From these data, 
the contrast was calculated individually for each measurement. 
Subsequently, 
the mean value and the corresponding standard deviation of $K$ were computed using \textit{numpy}~\cite{numpy}. 
The resulting averaged contrast values are visualized in Figure~\ref{fig:contrast}.
\begin{figure}[h] 
    \centering 
    \includegraphics[width=0.8\textwidth]{build/contrast.pdf} 
    \caption{Averaged contrast $K$ as a function of the polarization angle $\phi$ with a fit using the function described in Equation \eqref{eqn:K_ana}.} 
    \label{fig:contrast} 
\end{figure}
According to the theoretical description of the interferometer, 
the contrast is expected to follow the functional dependence
\begin{equation*}
    K_\text{theo} = 2\lvert \sin\phi \cos\phi \rvert,
\end{equation*}
which arises from the projection of the electric field components onto the analyzer axis.
In order to account for small misalignments of optical components and systematic deviations from the ideal setup, 
this expression is extended by introducing a phase offset $\delta$ and a scaling factor $K_0$.
The modified model function used for the fit is therefore given by
\begin{equation*}
    K(\phi) = 2K_0 \lvert \sin(\phi-\delta)\cos(\phi-\delta) \rvert.
    \label{eqn:K_ana}
\end{equation*}
The fit was performed using \textit{scipy}~\cite{scipy} and is shown together with the experimental data in Figure~\ref{fig:contrast}. 
The resulting fit parameters are
\begin{align*}
    K_0 &= 0.9914 \pm 0.0118, \\
    \delta &= (4.73 \pm 0.15)^\circ.
\end{align*}
The value of $K_0$ is close to unity, 
indicating a high-quality interference pattern and only minor losses in contrast due to imperfections of the optical components. 
The finite offset angle $\delta$ suggests a slight misalignment of the polarization optics with respect to the defined angular reference.

The maximum contrast was experimentally observed at a polarization angle of $\phi=\SI{50}{\degree}$. 
For all subsequent measurements, the polarization angle was therefore fixed at this value in order to ensure optimal interference visibility.



\subsection{The Refractive Index of Glass}
The refractive index of glass was determined by measuring the phase shift induced by rotating glass plates in one arm of the interferometer. 
As the plates are rotated, 
the optical path length changes, 
leading to a shift of the interference pattern. 
The number of intensity maxima $M$ passing through the center during a defined angular rotation provides a direct measure of this phase shift.
The glass plates were rotated by an angle of $\theta=\SI{10}{\degree}$, 
covering the angular interval from $30^\circ$ to $40^\circ$. 
This measurement was repeated ten times. 
The measured numbers of intensity maxima are listed in Table~\ref{tab:glas}.
\begin{table}
    \centering
    \begin{tabular}{SSS}
        \toprule
        {Measurement} & {$M$} & {$n$} \\
        \midrule
        1  & 29 & 1.431 \\ 
        2  & 32 & 1.498 \\
        3  & 35 & 1.572 \\
        4  & 35 & 1.572 \\
        5  & 34 & 1.546\\
        6  & 33 & 1.522\\
        7  & 33 & 1.522\\
        8  & 34 & 1.546\\
        9  & 32 & 1.498\\
        10 & 37 & 1.624\\
        \bottomrule       
    \end{tabular}
    \caption{Measured values of the number of intensity maxima $M$ that pass through the center as the glass plates are rotated by $10^\circ$. 
            Also the resulting refractive indices are included.} 
    \label{tab:glas}     
\end{table}

Since the glass plates are symmetrically tilted by $\theta_0=\pm\SI{10}{\degree}$, 
the general expression for the phase shift %(see Section~\ref{sec:phase_theory}) Passende Sectin angeben
can be simplified by setting $\theta_1=-\theta_2=\pm 10$, resulting in
\begin{equation*}
    \Delta\Phi(\theta) = 2\pi\frac{D}{\lambda_\text{vac}}\frac{n-1}{n}\cdot 2\theta_0\theta.
\end{equation*}
Using the relation between phase shift and the number of observed intensity maxima,
\begin{equation*}
    M = \frac{\Delta\Phi}{2\pi},
\end{equation*}
one obtains
\begin{equation}
    M = \frac{D}{\lambda_\text{vac}}\frac{n-1}{n}\cdot 2\theta_0\theta.
    \label{eqn:M_ana}
\end{equation}

Here, $\lambda_\text{vac}=\SI{632.99}{\nano\metre}$ denotes the wavelength of the laser, 
and $D=\SI{1}{\milli\metre}$ is the thickness of the glass plates~\cite{V64}. 
Solving Equation~\eqref{eqn:M_ana} for the refractive index yields
\begin{equation*}
    n = \frac{2\theta_0\theta D}{2\theta_0\theta D - \lambda_\text{vac} M}.
\end{equation*}
The refractive index calculated for each individual measurement is also included in Table~\ref{tab:glas}.

Using the values listed in Table~\ref{tab:glas}, the resulting mean refractive index of the glass is
\begin{equation}
    \bar{n} = \num{1.533 \pm 0.019}.
\end{equation}



\subsection{The Refractive Index of Gas}
As described in \autoref{sec:ind} the refractive index of a gas is measured by observing the phase shift induced by changing the pressure inside a gas cell placed in one of the two beams.
The measured values for the number of intensity maxima $M$ observed while varying the pressure from $\SI{0}{\milli\bar}$ to $\SI{1000}{\milli\bar}$ are listed in Table~\ref{tab:gas}.
\begin{table}
    \centering
    \begin{tabular}{S S S S}
        \toprule
        {$p/\si{\milli\bar}$} & {$M_1$} & {$M_2$} & {$M_3$}\\
        \midrule
        50   & 2  &  2 &  2 \\ 
        100  & 4  &  4 &  4 \\ 
        150  & 6  &  6 &  6 \\ 
        200  & 8  &  8 &  8 \\ 
        250  & 10 & 10 & 10 \\
        300  & 12 & 12 & 13 \\
        350  & 14 & 15 & 15 \\
        400  & 17 & 16 & 17 \\
        450  & 19 & 19 & 19 \\
        500  & 21 & 21 & 21 \\
        550  & 23 & 23 & 23 \\
        600  & 25 & 25 & 25 \\
        650  & 27 & 27 & 28 \\
        700  & 29 & 29 & 30 \\
        750  & 31 & 31 & 32 \\
        800  & 33 & 33 & 34 \\
        850  & 35 & 36 & 36 \\
        900  & 38 & 38 & 38 \\
        950  & 40 & 40 & 40 \\
        1000 & 42 & 42 & 42 \\
    \end{tabular}
    \caption{Measured values of the number of maxima $M$ that passed the center for different gas pressures $p$.}
    \label{tab:gas}
\end{table}
Because the values for the different measurements are very similar,
in figure~\ref{fig:gas} is only shown the value of $M_1$.
\begin{figure}
    \centering 
    \includegraphics[width=0.8\textwidth]{build/n_air.pdf}
    \caption{Number of maxima $M$ as a function of the gas pressure $p$ with a fit using the function described in Equation \eqref{eqn:M_gas}.}
    \label{fig:gas}
\end{figure}

In the case where $n \approx \num{1}$, the Lorentz-Lorenz law (\autoref{eqn:lll}) can be simplified to 
\begin{equation}
    n=\frac{3}{2}\frac{Ap}{RT}+1.
    \label{eqn:n_gas}
\end{equation}
With the measured room temperature $T_{\text{room}} = \SI{20.3}{\celsius}$ 
and $L = \SI{100}{\milli\metre}$, this results in a linear function 
\begin{equation}
    n(p)=\frac{3}{2}\frac{p}{RT_\text{room}}\cdot a+b
    \label{eqn:M_gas}
\end{equation}
where the parameters $a$ and $b$ are fitted to the data. 
The fit is shown in Figure \ref{fig:gas}. 
The values of the 
parameters are 
\begin{align*}
    a=(4.44 \pm 0.02)10^{-4} \quad\text{and}\quad b=0.9999987 \pm 0.0000008.
\end{align*}
Plugging these values into Equation \eqref{eqn:n_gas} with a 
temperature of $T = \SI{15}{\degreeCelsius}$ and a pressure of 
$p = \SI{1013}{\hecto\pascal}$ leads to the measured refractive 
index of air at standard atmospheric conditions:
\begin{equation*}
    n_\text{gas}=1.0002734\pm 0.0000015.
\end{equation*}
