\section{Discussion}
The contrast measurement yielded results that are consistent with theoretical expectations. 
The observed offset angle can be attributed either to a constant misalignment of the polarizer or to a systematic offset in the polarization angle of the laser source. 
The amplitude of the contrast function was found to be below its theoretical maximum, 
which indicates a non-ideal alignment of the interferometer setup.

The refractive index of the investigated glass sample was determined experimentally. 
Due to the wide variety of existing glass types and the lack of information regarding the specific composition of the sample, 
a direct comparison with literature values is not meaningful. 
Nevertheless, 
the measured refractive index lies within the typical range reported for common glass materials, 
indicating reasonable agreement with expected values.

The refractive index of the gas present in the laboratory under standard atmospheric conditions was also determined. 
While a small deviation from the literature value for air under standard conditions was observed, 
this difference is considered plausible. 
It can be attributed to variations in the exact gas composition in the room as well as to environmental factors such as humidity, 
which were not explicitly controlled during the measurement.
