\section{Experimental Procedure}
In this experiment a Sagnac interferometer is used for the measurements. An advantage of this type of interferometer compared to other types is that it operates particularly stable because both beams follow the same path. The following figure depicts the setup of the interferometer used:

\begin{figure}[h]
	\centering
	\includegraphics[width=0.7\textwidth]{plots/ifsetup.png}
	\caption{Setup of the used Sagnac interferometer \cite{V64}.}
	\label{fig:if}
\end{figure}

The coherent light source used in this experiment is a helium neon laser with a wavelength of $\lambda_\text{Laser} = \SI{632.99}{\nano\meter}$. This laser beam is reflected by the mirrors M1 and M2, 
passes through a polarizing filter, and is then split by the PBSC (= polarising beam-splitter cube).(, as 
described in Section ref{sec:theory}.) The polarization filter regulates the intensities of the s- and p-polarized beams emitted by the PBSC in a way that equal intensities for both beams are achieved at an angle of $\SI{45}{\degree}$ 
of the polarization filter. The beam are then reflected by mirrors Ma, Mb, and Mc. Both beams then converge again at the PBSC and form the output beam, which is split again so that the s- and p-polarized parts of the beam can be measured separately.

\subsection{Alignment of the interferometer}

The alignment of the interferometer is an essential part for the success of this experiment. At the start of the alignment process one has to center the laser beam on mirrors M1 and M2 to direct it through the center of the PBSC. In the next step of the alignment the adjustment plates has to be used to make the laser beam hit mirrors Ma and Mc as central as possible. Further alignments are used to make sure that both beams also hit mirror Mb central and converge there. Correct alignment is achieved if both beams rejoin at the PBSC. The quality of this alignment is shown by observing the interference pattern of the output beam, with the polarization filter set to $\SI{45}{\degree}$. If you can still see fringes in the pattern of the signal at this point, this shows phase differences of both laser beams and further alignment is necessary. 

\subsection{Determination of the contrast of the interferometer}

In the next step of the experiment, the contrast of the interferometer is measured to determine the maximum contrast.
In order to measure the contrast as a function of the polarization angle of the light, a glass plate holder that can be rotated is positioned in the beam path.
The position of the polarization filter is varied in \SI{10}{\degree} steps in a range from \SIrange{0}{180}{\degree}. 
The contrast of the interference pattern is finally determined by measuring the diode voltage as a function of the polarization angle.

\subsection{Measurement of the refractive indices}
For the measurement of the refractive index of glass a rotating glass plate is used in one arm of the interferometer. The rotation of the plate leads to a change of the optical path length, which also changes the interference pattern. The number of intensity maxima $M$ during a defined angular rotation of $\Theta$ = $\SI{10}{\degree}$ is counted. The glass plates are totally getting rotated by an angle of $\Theta$ = $\SI{10}{\degree}$, covering the angular interval from $\SI{30}{\degree}$ to $\SI{40}{\degree}$ and this measurement was repeated ten times. \newline

To measure the refractive index of air with the interferometer a gas capsule is put into the measurement way in the final part of the measurements. Here the  pressure inside of this gas capsule gets increased from a vacuum up to \SI{1000}{\milli\bar}. For this measurement the number of intensity minima and maxima is measured for every \SI{50}{\milli\bar}.