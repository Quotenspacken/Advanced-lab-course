\section{Durchführung}
Für die experimentelle Untersuchung der Amplituden- und Frequenzmodulation standen ein zweikanaliger hochfrequenter Signalgenerator, 
ein Oszilloskop, 
ein Spektrumanalysator sowie verschiedene diskrete Schaltungselemente zur Verfügung. 
Alle Schaltungen wurden so betrieben, 
dass die effektiven Spannungen $E_{\mathrm{eff}} = 2\mathrm{V}$ nicht überschritten wurden. 

Im ersten Schritt wurde ein Ringmodulator zur Erzeugung einer Amplitudenmodulation mit unterdrücktem Träger aufgebaut. 
Dazu wurden Träger- und Modulationssignal an die entsprechenden Eingänge gelegt und das Ausgangssignal auf dem Oszilloskop dargestellt. 
Der zeitliche Verlauf wurde als PNG gespeichert, 
zusätzlich wurde das zugehörige Spektrum mithilfe des Spektrumanalysators als CSV-Datei aufgenommen.

Anschließend wurde mithilfe der vorgegebenen Diodenschaltung der allgemeine Fall einer Amplitudenmodulation mit Trägerabstrahlung realisiert.
Das Ausgangssignal wurde sowohl zeitlich als PNG und CSV als auch im Frequenzbereich dokumentiert. 
Für die spätere Auswertung wurden die für den Modulationsgrad relevanten Signale vollständig aufgezeichnet.

Im nächsten Teil wurde die Schaltung zur Erzeugung eines schmalbandigen frequenzmodulierten Signals aufgebaut. 
Die Kombination aus Ringmodulator und $90^\circ$-Phasenschieber erzeugte am Ausgang ein FM-Signal, 
dessen zeitlicher Verlauf bei verschiedenen Trigger­einstellungen aufgenommen wurde. 
Es wurden sowohl Darstellungen des stabilen Bildes als auch des bei Eingangstriggerung entstehenden verschobenen Kurvenverlaufs als PNG gesichert.

Im Anschluss daran wurde ein phasenempfindlicher Gleichrichter mithilfe eines Ringmodulators aufgebaut, 
um die Demodulation amplitudenmodulierter Signale zu untersuchen. 
Dazu wurden ein Referenzträgersignal sowie das AM-Signal an die vorgesehenen Eingänge geführt. 
Am Ausgang des Tiefpasses wurde die Gleichspannung für verschiedene Phasenlagen gemessen und die Messreihe vollständig dokumentiert.

Daraufhin wurde ein amplitudenmoduliertes Signal mit Trägerabstrahlung erzeugt, 
wobei das Ausgangssignal eines Ringmodulators über einen Leistungsteiler mit einem Anteil des Trägersignals überlagert wurde. 
Die Demodulation erfolgte erneut über einen Ringmodulator mit nachgeschaltetem Tiefpass. 
Der zeitliche Verlauf des Signals hinter dem Ringmodulator sowie am Tiefpassausgang wurde jeweils als PNG gespeichert.

Als weiteres Verfahren wurde eine Hüllkurvendemodulation durchgeführt. 
Dazu wurde das amplitudenmodulierte Signal gleichgerichtet und anschließend über einen Tiefpass geführt.
Der zeitliche Verlauf vor und nach dem Tiefpass wurde jeweils aufgezeichnet, 
um die nachfolgende Auswertung zu ermöglichen.

Zum Abschluss wurde die Demodulation eines frequenzmodulierten Signals umgesetzt. 
Das FM-Signal wurde hierfür zunächst einem abgestimmten LC-Schwingkreis zugeführt,
um es in ein amplitudenmoduliertes Signal zu überführen.
Anschließend erfolgte die Demodulation über Gleichrichter und Tiefpass. 
Die Signale hinter dem Schwingkreis, 
hinter dem Gleichrichter und am Tiefpassausgang wurden jeweils separat als PNG gespeichert.
