\section{Modulation}
Nachrichtentechnische Systeme dienen der Übertragung von Informationen zwischen einem Sender und einem Empfänger und sind heute in einer Vielzahl technischer Anwendungen verbreitet. 
Dazu zählen klassische analoge Verfahren wie das Telefon oder der Rundfunk, 
aber auch moderne digitale Systeme wie ISDN, 
Mobilfunk oder Breitbandkabelnetze \cite{book}.
Damit ein zu übertragendes Nutzsignal geeignete Eigenschaften für die Übertragung besitzt, 
muss es in ein Sendesignal umgewandelt werden. 
Dieser Vorgang wird als \emph{Modulation} bezeichnet. 
Sie ermöglicht eine effiziente, 
robuste und reichweitenstarke Übertragung.
Ein ursprüngliches Nutzsignal ist häufig zu niederfrequent oder besitzt ungünstige spektrale Eigenschaften, 
um direkt ausgesendet zu werden. 
Durch die Modulation wird es auf eine hochfrequente Trägerwelle aufgesetzt, 
deren Parameter (Amplitude, Frequenz, Phase) durch das Nutzsignal verändert werden. 
Dadurch ergeben sich mehrere Vorteile:
\begin{itemize}
	\item Das Signal kann technisch leichter über Antennen abgestrahlt werden.
	\item Mehrere Kommunikationskanäle können im selben Frequenzraum genutzt werden (Frequenzmultiplex).
	\item Die Übertragung wird weniger störanfällig, da die hochfrequente Trägerwelle weniger anfällig für Verzerrungen und Rauscheinflüsse ist.
\end{itemize}

Im Folgenden werden die Amplituden- und Frequenzmodulation als grundlegende analoge Modulationsverfahren betrachtet.



\subsection{Amplitudenmodulation}
Bei der Amplitudenmodulation (AM) wird die Amplitude des hochfrequenten Trägersignals $s_T$ durch das niederfrequente Nutzsignal $s(t)$ verändert. 
Die Trägerfunktion lautet \cite{book}:
\begin{equation}
	s_T = [a_T + m s(t)] \cos(w_T t)
	\label{AM}
\end{equation}
Der Modulationsgrad $m$ beschreibt dabei die Stärke der Amplitudenänderung. 
Für ein sinusförmiges Nutzsignal
\begin{equation}
	s(t) = a_s\cos(w_st)
\end{equation}
ergibt sich
\begin{equation}
	a_T\cos(w_Tt) + \frac{ma_s}{2}\cos(w_t - w_s)t + \frac{ma_s}{2}\cos(w_T + w_s)t.
	\label{eq:AM}
\end{equation}
Das modulierte Signal besteht somit aus dem unmodulierten Träger (erster Summand), 
einem unteren Seitenband bei der Frequenz $w_T - w_s$ sowie einem oberen Seitenband bei $w_T + w_s$. 
Die beiden Seitenbänder enthalten die eigentliche Information des Nutzsignals.
\begin{figure}[h]
	\centering
	\includegraphics[width=0.7\textwidth]{Bilder/1}
	\caption{Amplitudenmoduliertes Signal (dünne Linie) mit eingezeichnetem Sendesignal (dicke Linie) in der Zeitdarstellung \cite{Anleitung}.}
	\label{fig:AM}
\end{figure}
Aus den Extremwerten $U_{\max}$ und $U_{\min}$ in Abb.~\ref{fig:AM} lässt sich der Modulationsgrad bestimmen:
\begin{equation*}
	m = \frac{U_{max} - U_{min}}{U_{max} + U_{min}}
\end{equation*}
Auch im Frequenzbereich lässt sich die AM leicht analysieren. 
In Abb.~\ref{fig:AMW} erkennt man Trägerfrequenz sowie oberes und unteres Seitenband.
\begin{figure}[h]
	\centering
	\includegraphics[width=0.7\textwidth]{Bilder/2}
	\caption{Darstellung der Amplitudenmodulation im Frequenzbereich \cite{Wikipedia}.}
	\label{fig:AMW}
\end{figure}
Bei nicht-harmonischen Nutzsignalen treten in den Seitenbändern zusätzliche Frequenzanteile auf, 
welche das Spektrum verbreitern. 
Der Modulationsgrad kann im Frequenzbereich über das Verhältnis zwischen Seitenbandamplitude $U_{TM}$ und Trägeramplitude $U_T$ bestimmt werden:
\begin{equation*}
	m = \frac{U_{TM}}{U_T}.
\end{equation*}



\subsection{Frequenzmodulation}
\label{kap:fmodulation}
Bei der Frequenzmodulation (FM) wird die Momentanfrequenz der Trägerwelle durch das Nutzsignal verändert:
\begin{equation*}
	w(t) = w_T + ms(t)
\end{equation*}
Die resultierende zeitabhängige Spannung lautet:
\begin{equation}
	U(t) = U\sin(w_Tt+m\frac{w_T}{w_S}\cos(w_S)t).
	\label{FM}
\end{equation}
Unter der Annahme einer Schmalband-Frequenzmodulation ($m\frac{w_T}{w_S} \ll 1$) lässt sich Gleichung~\ref{FM} mithilfe eines Additionstheorems und einer Kleinwinkelnäherung vereinfachen. 
Dadurch wird eine strukturelle Ähnlichkeit zur Amplitudenmodulation gemäß Gleichung~\ref{eq:AM} sichtbar:
\begin{equation*}
	a_T\cos(w_Tt) + \frac{ma_s}{2}\sin(w_t - w_s)t + \frac{ma_s}{2}\sin(w_T + w_s)t.
\end{equation*}
Die Frequenzmodulation unterscheidet sich in diesem Grenzfall lediglich durch einen Phasenversatz von $90^\circ$.
\begin{figure}[h]
	\centering
	\includegraphics[width=0.7\textwidth]{Bilder/3}
	\caption{Frequenzmoduliertes Signal (dünne Linie) mit eingezeichnetem Modulationssignal (dicke Linie) \cite{book}.}
	\label{fig:FM}
\end{figure}
Bei stärkerer Frequenzmodulation ist die Kleinwinkelnäherung nicht mehr zulässig. 
Analog zur AM entstehen dann weitere spektrale Linien, 
die sich über mehrere sogenannte \emph{Seitenordnungen} erstrecken. 
Dies führt zu einem breiteren Frequenzspektrum, 
wie es beispielsweise beim UKW-Rundfunk genutzt wird.





\section{Modulations- und Demodulationsverfahren}
\subsection{Modulation}
Zur Erzeugung eines amplitudenmodulierten Signals wird ein nichtlinearer Baustein benötigt, 
der ein Produkt aus Trägerspannung und Nutzspannung erzeugt. 
Eine einfache Realisierung erfolgt durch eine Diode, 
wie in Abb.~\ref{fig:Diode} dargestellt.
\begin{figure}[h]
	\centering
	\includegraphics[width=0.5\textwidth]{Bilder/4}
	\caption{Erzeugung einer Amplitudenmodulation mit einer Diode \cite{Anleitung}.}
	\label{fig:Diode}
\end{figure}
Die vereinfachte Kennlinie einer Diode lautet \cite{Anleitung}:
\begin{equation*}
	I(U)\approx a_0+a_1U+a_2U^2.
\end{equation*}
Der quadratische Term liefert dabei den zur Modulation notwendigen Produktanteil.

Eine präzisere AM mit unterdrücktem Träger kann mit einem Ringmodulator erzeugt werden, 
siehe Abb.~\ref{fig:FMR}. 
Hier wird der Träger effizient ausgeblendet, 
sodass nur die Seitenbänder verbleiben. 
Dies erleichtert die spätere Demodulation und führt zu einer besseren Störfestigkeit.
\begin{figure}[h]
	\centering
	\includegraphics[width=0.7\textwidth]{Bilder/5}
	\caption{Erzeugung einer Amplitudenmodulation mit Trägerunterdrückung durch einen Ringmodulator \cite{Anleitung}.}
	\label{fig:FMR}
\end{figure}

Zur Erzeugung frequenzmodulierter Signale lässt sich ebenfalls ein Ringmodulator verwenden. 
Mithilfe eines Phasenschiebers kann aus einem amplitudenmodulierten Signal ein schmalbandig frequenzmoduliertes Signal erzeugt werden, 
wie in Abb.~\ref{fig:fm} dargestellt.
\begin{figure}[h]
	\centering
	\includegraphics[width=0.7\textwidth]{Bilder/6}
	\caption{Schaltung zur Erzeugung eines frequenzmodulierten Signals mit geringem Frequenzhub \cite{Anleitung}.}
	\label{fig:fm}
\end{figure}



\subsection{Demodulation}
Für die Rückgewinnung des Nutzsignals existieren verschiedene Demodulationsmethoden.
Eine einfache Möglichkeit für AM ist das Gleichrichten des modulierten Signals und anschließendes Filtern mittels eines Tiefpasses (Hüllkurvendemodulator). Die entsprechende Schaltung ist in Abb.~\ref{fig:Demod1} gezeigt.
\begin{figure}[h]
	\centering
	\includegraphics[width=0.7\textwidth]{Bilder/7}
	\caption{Demodulation mithilfe eines Gleichrichters und eines Tiefpasses \cite{Anleitung}.}
	\label{fig:Demod1}
\end{figure}

Eine alternative Methode ist die Demodulation mittels Ringmodulator (synchrone Demodulation), 
siehe Abb.~\ref{fig:Demod2}. 
Diese Methode ist besonders für Signale mit unterdrücktem Träger geeignet und ermöglicht eine präzisere Rekonstruktion des Nutzsignals, 
da der Demodulator den ursprünglichen Träger wieder einspeist.
\begin{figure}[h]
	\centering
	\includegraphics[width=0.7\textwidth]{Bilder/8}
	\caption{Demodulation mithilfe eines Ringmodulators \cite{Anleitung}.}
	\label{fig:Demod2}
\end{figure}

Für die Demodulation eines frequenzmodulierten Signals wird zunächst ein Schwingkreis verwendet, 
der das FM-Signal aufgrund seiner frequenzabhängigen Amplitudenübertragung in ein amplitudenmoduliertes Signal umwandelt. 
Dieses kann anschließend wie bei der AM durch Gleichrichter und Tiefpass demoduliert werden (Abb.~\ref{fig:Demod3}).
\begin{figure}[h]
	\centering
	\includegraphics[width=0.7\textwidth]{Bilder/9}
	\caption{Demodulation eines frequenzmodulierten Signals mittels Schwingkreis und anschließender AM-Demodulation \cite{Anleitung}.}
	\label{fig:Demod3}
\end{figure}









