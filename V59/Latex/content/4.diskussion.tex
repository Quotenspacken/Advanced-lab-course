\section{Diskussion}
Bei diesem Versuch wurden mehrere Modulations- und demodulationsverfahren erfolgreich durchgeführt. \newline
\noindent
Als Erstes wurde ein amplitudenmoduliertes Signal mit Hilfe eines Ringmodulators erzeugt.  Wie erwartet konnte als Ausgangssignal über das Oszilloskop eine Schwebung aufgenommen werden.

Als nächstes Modulationsverfahren wurde eine Amplitudenmodulation mit Trägerabstrahlung realisiert. Hierzu wurde eine Schaltung nach \autoref{fig:Diode} aufgebaut und das entsprechende Ausgangssignal mit dem Oszilloskop aufgenommen. Dabei wurde auch der Modulationsgrad bestimmt. Als Ergebnis für den Modulationsgrad wurde $m \approx 0.76$ bestimmt. Dieses Ergebnis liegt im für den Modulationsgrad erwarteten Bereich. 

Danach wurde ein frequenzmoduliertes Signal erzeugt. Die dafür genutzte Schaltung ist in \autoref{fig:fm} dargestellt. Durch Eingangstriggerung mit den passenden Triggereinstellungen konnte eine maximal verschmierte Signalkurve erzeugt und aufgenommen werden. Zu dem hier erzeugten frequenzmodulierten Signal wurde ein Modulationsgrad von $m \approx 0.65$ bestimmt.

Ein durchgeführtes Demodulationsverfahren ist die Demodulation eines amplitudenmodulierten Signals mit Ringmodulator. Dazu wurde die Schaltung eines phasenempfindlichen Gleichrichters mit Ringmodulator und Tiefpass aufgebaut. Hinter dem Tiefpass konnten Signale für verschiedene Phasenverzögerungen aufgenommen werden.

Bei der nächsten Messung wurde ein amplitudenmoduliertes Signal mit Trägerabstrahlung durch eine Schaltung bestehend aus einem Ringmodulator und einem Tiefpass demoduliert. Die erfolgreiche Demodulation hat sich hier dadurch gezeigt, dass hinter dem Tiefpass eine phasenverschobene Kopie des Modulationssignals mit kleinerer Amplitude aufgenommen wurde.

Als weiteres Demodulationsverfahren eines amplitudenmodulierten Signals wurde die Demodulation des Signals mit Hilfe eines Gleichrichters und eines Tiefpasses durchgeführt. Auch bei dieser Messung konnte den Erwartungen entsprechend am Ausgang des Tiefpasses eine phasenverschobene Kopie des Modulationssignals aufgenommen werden.

Schließlich wurde die Demodulation eines frequenzmodulierten Signals realisiert. Dabei wurde das frequenzmodulierte Signal über einen Schwingkreis in ein amplitudenmoduliertes Signal umgewandelt. Der zeitliche Verlauf des amplitudenmodulierten Signals hinter dem Schwingkreis konnte aufgezeichnet werden. Anschließend wurde das amplitudenmodulierte Signal durch einen Gleichrichter und einen Tiefpass demoduliert. Wie erwartet konnte auch hier hinter dem Tiefpass eine phasenverschobene Kopie des Modulationssignals aufgenommen werden.\newline
\noindent
Im Verlauf des Versuchs konnten also verschiedene Verfahren zur Modulation und zur Demodulation realisiert werden. Insgesamt entsprechen die aufgenommenen Signale und die erhaltenen Messwerte den Erwartungen.