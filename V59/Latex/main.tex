\documentclass[
  bibliography=totoc,     % Literatur im Inhaltsverzeichnis
  captions=tableheading,  % Tabellenüberschriften
  titlepage=firstiscover, % Titelseite ist Deckblatt
]{scrartcl}

% Language and Encoding
\usepackage[utf8]{inputenc}
\usepackage[T1]{fontenc}
%\usepackage[english]{babel}
\usepackage[ngerman]{babel}

%Font
\usepackage{newtxtext,newtxmath}

% Figures, Graphics, and Tables
\usepackage[pdftex,dvipsnames,table]{xcolor}
\usepackage[pdftex]{graphicx}
\usepackage{booktabs}
\usepackage{longtable}
\usepackage{float}

% Math
\usepackage{mathtools}
\usepackage{bm}
\usepackage{xfrac}
\usepackage[inference]{semantic}
\usepackage{witharrows}
\usepackage[amsmath,thmmarks,hyperref]{ntheorem}

% Miscellaneous
\usepackage{csquotes}
\usepackage{enumitem}
\usepackage{xurl}
\usepackage{setspace}
\usepackage{mdframed}
\usepackage{bbding}

% Bibliography and Glossary
\usepackage[backend=biber,style=numeric,sorting=none]{biblatex}
\bibliography{lib.bib}

\usepackage{glossaries}

\usepackage{pdfpages}

% Physical Units and Numbers
\usepackage[locale=DE]{siunitx}

% References
\usepackage{hyperref}

%Titelseite
\author{%
  Philip Jaletzky\\%
  \href{mailto:philip.jaletzky@tu-dortmund.de}{philip.jaletzky@tu-dortmund.de}%
  \and%
  Sebastian Rüßmann\\%
  \href{mailto:sebastian.ruessmann@tu-dortmund.de}{sebastian.ruessmann@tu-dortmund.de}%
}
\publishers{TU Dortmund – Fakultät Physik}

\subject{V59}
\title{Modulation und Demodulation}
\date{%
  Ausführung: 17. November 2025
  \hspace{3em}
  Abgabe: \today
}

\begin{document}

\maketitle
\thispagestyle{empty}
\tableofcontents
\newpage

\section{Theorie}
\subsection{Kosmische Myonen}
Myonen sind Elementartteilchen, die zur Familie der Leptonen gehören.
Sie besitzen eine negative elektrische Ladung von \(-1e\) und eine Ruheenergie von etwa \(105{,}7\,\mathrm{MeV}/c^2\), was ungefähr dem 207-fachen der Elektronenmasse entspricht.
Neben Teilchenbeschleunigern ist die kosmische Strahlung eine bedeutende Quelle für Myonen.
Hier entstehen aus hochenergetischen Primärteilchen, 
die in die Erdatmosphäre eindringen, 
durch Kollisionen mit Atomkernen sekundäre Teilchen wie Pionen.
Pionen zerfallen in den meisten fällen in Myonen und Myon-Neutrinos gemäß der Reaktionen:
\begin{align*}
    \pi^+ &\rightarrow \mu^+ + \nu_\mu \quad \\
    \pi^- &\rightarrow \mu^- + \bar{\nu}_\mu. 
\end{align*}
Anders als Elektronen sind Myonen instabil und zerfallen nach einer durchschnittlichen Lebensdauer von etwa \(2{,}2\,\mu\mathrm{s}\) \cite{griff}.
Der hüfigste Zerfallskanal ist der in ein Elektron, ein Elektron-Antineutrino und ein Myon-Neutrino:
\begin{equation*}
    \mu^- \rightarrow e^- + \bar{\nu}_e + \nu_\mu.      
\end{equation*}



\section{Definition der Lebensdauer}\label{sec:tau}
Die Lebensdauer \(\tau\) eines instabilen Teilchens ist definiert als der Mittelwert der Zeitintervalle zwischen der Entstehung und dem Zerfall des Teilchens.
Da jedes Teilchen eine individuelle Lebensdauer hat,
ist dieser Prozess statistisch Verteilt.
Diese Verteilung lässt sich durch die Zerfallskonstatnte \(\lambda\) charakterisieren, 
die den Zerfall pro Zeiteinheit beschreibt.
Im infinitesimalen Zeitintervall \(\mathrm{d}t\) ist die Wahrscheinlichkeit \(\mathrm{d}W\), dass ein Teilchen zerfällt, proportional zur Zerfallskonstanten und dem Beobachtungszeitintervall \(\mathrm{d}t\):
\begin{equation*}
    \mathrm{d}W = \lambda \, \mathrm{d}t.           
\end{equation*}
Da nun Teilchen unabhängig voneinander zerfallen und die Zerfallswahrscheinlichkeit nicht vom Alter des Teilchens abhängt, 
ergibt sich für die Zahl
\(N(t)\) der noch nicht zerfallenen Teilchen zum Zeitpunkt \(t\):
\begin{equation}
    dN = -N(t) \, \mathrm{d}W = -\lambda N(t) \, \mathrm{d}t.
    \label{eq:dN}
\end{equation}
Wenn N eine große Zahl ist,
dann lässt sich Gleichung \eqref{eq:dN} näherungsweise Integrieren.
Bildet man zussätzlich daraus die Verteilungsfunktion der Lebensdauer $t$,
so ergibt sie sich zu:
\begin{equation}
    \frac{dN(t)}{N_0} = -\lambda e^{-\lambda t} \, \mathrm{d}t,
    \label{eq:verteilung}
\end{equation}
der Exponentielle Verteilung mit $N_0$ als der Gesamtzahl der betrachteten Teilchen.
Um nun die charakteristische Lebensdauer \(\tau\) zu bestimmen,
wird der Mittelwert aller möglichen Lebensdauern berechnet:
\begin{equation}
    \tau = \int_0^\infty \lambda t e^{-\lambda t} \, \mathrm{d}t = \frac{1}{\lambda}.
    \label{eq:lebensdauer}
\end{equation}  

\subsection{Technik für Teilchenphysik}
\subsection{Photomultiplier}
Photomultiplier (PMTs) sind hochempfindliche Detektoren,
die einzelne Photonen in elektrische Signale umwandeln und verstärken können.       
Ein PMT besteht aus einer Photokathode,
die Photonen absorbiert und Elektronen emittiert.
Dabei tragend ist der sogenannte Photoeffekt.
Die emittierten Elektronen werden dann durch eine Reihe von Dynoden beschleunigt,       
die jeweils eine höhere Spannung haben als die vorherige.
Wenn ein Elektron auf eine Dynode trifft,
schlägt es mehrere Elektronen heraus,
dadurch wird die anfangs geringe menge an Elektronen vervielfacht \cite{Leo}.

\subsubsection{Szintillationsdetektoren}
Szintillationsdetektoren sind Geräte, die ionisierende Strahlung durch die Emission von Lichtblitzen nachweisen.
Sie bestehen aus einem Szintillationsmaterial, 
das bei der Wechselwirkung mit geladenen Teilchen Licht emittiert.
Dieses Licht kann dann von einem Photomultiplier (PMT) detektiert werden, 
der die Lichtsignale in elektrische Signale umwandelt.

Man unterscheidet zwischen organischen und anorganischen Szintillatoren,
die für unterschiedliche Anwendungen optimiert sind.
Organische Szintillatoren, wie z.B. Kunststoff- oder Flüssigszintillatoren,
reagieren schnell auf ionisierende Strahlung und eigen sich daher gut für Experimente,
die eine hohe Zeitauflösung erfordern.
Anorganische Szintillatoren, wie z.B. Natriumiodid (NaI) oder Bismutgermanat (BGO),
besitzen eine hohe Energieauflösung und sind besonders geeignet für die Untersuchung von hochenergetischen Teilchen\cite{Leo}.

\subsection{Diskriminatoren}
Diskriminatoren sind elektronische Bauelemente,
die Signale nach ihrer Amplitude filtern.
Sie geben nur dann ein Ausgangssignal ab,
wenn das Eingangssignal einen bestimmten Schwellenwert überschreitet.
Dadurch kann zum Beispiel das Rauschen unterdrückt werden.

\subsection{Multikanal-Analysatoren}
Multikanal-Analysatoren (MCA) sind Geräte,
die Signale in verschiedene Kanäle aufteilen und analysieren können.    
Sie werden häufig in der Teilchenphysik und Kernphysik eingesetzt,
um die Energieverteilung von Teilchen zu messen.
Ein MCA besteht aus einem Analog-Digital-Wandler (ADC),
der das analoge Eingangssignal in digitale Werte umwandelt.
Diese digitalen Werte werden dann in verschiedene Kanäle sortiert,
die jeweils eine bestimmte Energie oder Amplitude repräsentieren \cite{Leo}.
\section{Experimental Procedure}
The Thorlabs Quantum Cryptography Demonstration Kit (EDU-QCRY1)\cite{thor} provides the optical components needed for this experiment. The experimental procedure is divided into two parts. First the standard BB84 protocol will be executed and in the second step the BB84 procedure with decoy states simulated by different wavelengths will be tested.

\subsection{Standard BB84 procedure}
In the first part of the experiment the standard version of the BB84 procedure without decoy states gets executed. First Alices laser has to be aligned with Bobs detectors using alignment tools or by overlapping the beams directly. The laser should operate continuously in adjustment mode (yellow LED). Once aligned it has to be switched to pulse mode (green LED). After this test measurements are performed, which confirm that the polarization states are correct. All eight possible cases must be verified before starting the experiment.\\
After the alignment the key generation can be started. Both lasers and detectors have to be set to measurement mode (green LEDs). Alice randomly generates 52 bits together with random bases (+ or x) for encoding. Bob independently generates his own random bases for each measurement and all of the 52 raw detection outcomes are recorded. During the sifting, Alice and Bob compare their bases information. They keep only those measurements where their bases match.\\
\noindent
For the error testing and to estimate noise or eavesdropping effects, Alice and Bob compare a small subset of their sifted bits. If mismatches occur between their bits, these count as errors. These errors are used to calculate the Quantum Bit Error Rate (QBER):\\
\begin{equation}
\text{QBER}=  \frac{N_{\text{errors}}}{N_{\text{sifted bits}}} 
\end{equation}
\\
\noindent
A high QBER is a sign for interference from an eavesdropper or noise.\\

\noindent
In the next step of the experiment Eve is getting introduced. Eves module has to be inserted between Alices and Bobs modules as it can be seen in \autoref{fig:setup}. For Eve you also has to randomly select measurement bases for the interception of Alices messages. After this the message has to be further transmitted to Bob using the same basis. This process typically increases the error rate by about 25$\%$, showing the effects of an eavesdropper attack.\\
Finally the sifted key has to be used as a one-time pad to encrypt a short message from Alice to Bob. After decryption on Bobs side, correctness confirms successful secure communication under BB84 conditions.\\

\noindent
The described setup with the three parties Alice (Sender), Eve (Eavesdropper) and Bob (Receiver) can be seen in \autoref{fig:setup} \cite{thor}.

\begin{figure}[h]
    \centering
    \includegraphics[width=\textwidth]{plots/setup.png}
    \caption{The standard experimental setup for the BB84 procedure \cite{thor}.}
    \label{fig:setup}
\end{figure}

\subsection{BB84 procedure with decoy states}
In the second part of the experiment the procedure gets extended by using two different wavelengths for the messages to simulate the decoy state method.\\
\noindent
The alignment procedure has to be repeated for both lasers (red and green wavelengths). Here also all eight cases of polarization settings have to be verified before starting key generation.
In this extended version of the protocol information is encoded using trits (0,1,2) and two different wavelengths, red (r) and green (g). Specific wavelength–basis combinations correspond to signal states used for key generation: \\
\noindent
For basis [+]: 0=r0+, 1=r1+, 2=g0+ \\
For basis [×]: 0=g0x, 1=g1x, 2=r0x \\
\noindent
Certain combinations like r1x ([x] basis) and g1+ ([+] basis) act as decoy states that should never appear if no interference occurs.\\
\noindent
Next Alice and Bob again generate random sequences of 52 trits/bases each and record their detection outcomes accordingly. As before, Alice and Bob exchange their basis information but retain only matching cases for further analysis. Then they compare all matched values from the sifted dataset to determine:

\begin{itemize}
	\item $N_{\text{matched sifted bits}}$: Number of correct/matched detection outcomes
	\item $N_{\text{errors}}$: Number of errors
	\item $N_{\text{decoys}}$: Number of detected decoy states
\end{itemize}
\noindent
With these results two quantities can then be computed:

\begin{align}
\text{Quantum Bit Error Rate (QBER)} &= \frac{N_{\text{errors}}}{N_{\text{matched sifted bits}}} \\
\text{Decoy fraction} &= \frac{N_{\text{decoys}}}{N_{\text{matched sifted bits}}}
\end{align}
\noindent
In the next step Eve is inserted in the Decoy setup. The Eve module gets inserted between Alice and Bob as it can be seen in \autoref{fig:extinction}. The eight cases from the alignment step also has to be tested again.
\begin{figure}[h]
    \centering
    \includegraphics[width=\textwidth]{plots/secondwl.png}
    \caption{The extended experimental setup with the Alice, Eve and Bob modules modified to add a second wavelength \cite{cry}. (HWP: half-wave plate, PBS: polarizing beam splitter)}
    \label{fig:extinction}
\end{figure}\\
\noindent
Then Eve again intercepts Alices pulses using randomly chosen bases, measures the corresponding bit (0 = 0° for [+] basis, -45 ° for [x]; 1 = 90° for [+] basis, 45° for [x]) and then transmits them in the same way further to Bob. \\
\noindent
After this measurements the error testing procedure has to be repeated under these conditions. Then the results should be compared with the baseline measurements, eventually revealing how eavesdropping affects both the Quantum Bit Error Rate and the decoy state statistics.
\section{Auswertung}
Bei diesem Versuch wurden verschiedene Verfahren der Amplituden- und Frequenzmodulation in mehreren Teilmessungen untersucht. Die Durchführung der einzelnen Verfahren und Messungen entspricht der Beschreibung in \autoref{sec:df}. Bei jedem dieser Messvorgänge wurden sowohl die Modulationsamplitude $U_\text{M}$ und die Modulationsfrequenz $f_\text{M}$, als auch die Trägeramplitude $U_\text{T}$ und die Trägerfrequenz
$f_\text{T}$ anhand der gemessenen Signale dokumentiert. Im Folgenden werden zu jeder der Teilmessungen diese entsprechenden Werte genannt und es werden die jeweils erhaltenen Signale dargestellt.

\subsection{Erzeugung und Analyse modulierter Signale}

Zu Beginn wurde, wie in den vorherigen Kapiteln beschrieben, eine  amplitudenmodulierte Schwingung mit Trägerunterdrückung mit Hilfe der Schaltung aus \autoref{fig:FMR} durch einen Ringmodulator  erzeugt. Dabei wurden mehrere Messungen durchgeführt, um verschiedene Ausgangssignale aufnehmen zu können. Die so entstandenen Schwebungen sind in den folgenden Abbildungen dargestellt. \newline 
Das Ergebnis der ersten dieser Messungen ist in \autoref{fig:amplModOszi} zu sehen.

\begin{figure}[H]
  \centering
  \includegraphics[width=.8\textwidth]{Bilder/osz/scope_0.png}
  \caption{Die erste gemessene Schwebung entstanden bei der mit Ringmodulator erzeugten amplitudenmodulierten Schwingung mit Trägerunterdrückung.}
  \label{fig:amplModOszi}
\end{figure}

Bei der Messung zu \autoref{fig:amplModOszi} wurden für die Trägeramplitude und die Trägerfrequenz
\begin{align*}
  f_\text{M} &= \SI{10}{\kilo\hertz} & U_\text{M} &= \SI{500}{\milli\volt}\\
  f_\text{T} &= \SI{500}{\kilo\hertz} & U_\text{T} &= \SI{1}{\volt}
\end{align*}
bestimmt.

\begin{figure}[H]
  \centering
  \includegraphics[width=.8\textwidth]{Bilder/osz/scope_1.png}
  \caption{Die zweite gemessene Schwebung entstanden bei der mit Ringmodulator erzeugten amplitudenmodulierten Schwingung mit Trägerunterdrückung.}
  \label{fig:amplModOszi11}
\end{figure}

Die zu \autoref{fig:amplModOszi11} bestimmten Werte sind 
\begin{align*}
  f_\text{M} &= \SI{10}{\kilo\hertz} & U_\text{M} &= \SI{500}{\milli\volt}\\
  f_\text{T} &= \SI{100}{\kilo\hertz} & U_\text{T} &= \SI{1}{\volt}.
\end{align*}

\begin{figure}[H]
  \centering
  \includegraphics[width=.8\textwidth]{Bilder/osz/scope_2.png}
  \caption{Die dritte gemessene Schwebung entstanden bei der mit Ringmodulator erzeugten amplitudenmodulierten Schwingung mit Trägerunterdrückung.}
  \label{fig:amplModOszi12}
\end{figure}

Aus dem Modulations- und dem Trägersignal zu \autoref{fig:amplModOszi12} wurden 
\begin{align*}
  f_\text{M} &= \SI{10}{\kilo\hertz} & U_\text{M} &= \SI{500}{\milli\volt}\\
  f_\text{T} &= \SI{100}{\kilo\hertz} & U_\text{T} &= \SI{1}{\volt}
\end{align*}
ermittelt.

\begin{figure}[H]
  \centering
  \includegraphics[width=.8\textwidth]{Bilder/osz/scope_3.png}
  \caption{Die vierte gemessene Schwebung entstanden bei der mit Ringmodulator erzeugten amplitudenmodulierten Schwingung mit Trägerunterdrückung.}
  \label{fig:amplModOszi13}
\end{figure}

Die zu \autoref{fig:amplModOszi13} bestimmten Werte sind 
\begin{align*}
  f_\text{M} &= \SI{20}{\kilo\hertz} & U_\text{M} &= \SI{100}{\milli\volt}\\
  f_\text{T} &= \SI{500}{\kilo\hertz} & U_\text{T} &= \SI{500}{\milli\volt}.
\end{align*}

%\newline
\noindent
Bei dem nächsten Messverfahren wurde die Amplitudenmodulation mit Trägerabstrahlung realisiert. Dazu wurde eine Schaltung nach \autoref{fig:Diode} genutzt.

\begin{figure}[H]
  \centering
  \includegraphics[width=.8\textwidth]{Bilder/osz/scope_4.png}
  \caption{Amplitudenmoduliertes Signal mit Trägerabstrahlung.}
  \label{fig:amplModOszi14}
\end{figure}

Die zu \autoref{fig:amplModOszi14} bestimmten Werte sind 
\begin{align*}
  f_\text{M} &= \SI{200}{\kilo\hertz} & U_\text{M} &= \SI{500}{\milli\volt}\\
  f_\text{T} &= \SI{5}{\mega\hertz} & U_\text{T} &= \SI{1}{\volt}.
\end{align*}

Zu dieser Messung wurde zusätzlich der Modulationsgrad m bestimmt. Wie in \autoref{sec:mod} beschrieben, kann der Modulationsgrad unter anderem nach \autoref{eq:m} bestimmt werden. Da zu dieser Messung auch eine CSV-Datei aufgenommen wurde, konnten die Signale gefittet werden. So wurden die benötigten Amplituden aus den Daten zu 
\begin{align*}
  U_\text{max} &= \SI{23.89}{\milli\volt} & U_\text{min} &= \SI{3.3}{\milli\volt}
\end{align*}
bestimmt.
Für den Modulationsgrad ergibt dies also:
\begin{equation*}
	m = \frac{\SI{23.89}{\milli\volt} - \SI{3.3}{\milli\volt}}{\SI{23.89}{\milli\volt} + \SI{3.3}{\milli\volt}} \approx 0.76
\end{equation*}



Anschließend wurde eine Schaltung nach \autoref{fig:fm} aufgebaut, um ein frequenzmoduliertes Signal zu erzeugen. Mit der Kombination aus Ringmodulator und $90^\circ$-Phasenschieber konnte folgendes Signal aufgenommen werden:

\begin{figure}[H]
  \centering
  \includegraphics[width=.8\textwidth]{Bilder/osz/scope_5.png}
  \caption{Durch die Kombination aus Ringmodulator und $90^\circ$-Phasenschieber erzeugtes Signal mit der durch Eingangstriggerung entstehenden verschobenen Kurve.}
  \label{fig:amplModOszi1}
\end{figure}

Der Bildausschnitt der Signalkurve auf dem Oszilloskop wurde so eingestellt, dass die maximale Verschmierung sichtbar wird, wie in \autoref{fig:amplModOszi1} zu sehen ist.
Die zu \autoref{fig:amplModOszi1} bestimmten Werte sind 
\begin{align*}
  f_\text{M} &= \SI{200}{\kilo\hertz} & U_\text{M} &= \SI{500}{\milli\volt}\\
  f_\text{T} &= \SI{5}{\mega\hertz} & U_\text{T} &= \SI{1}{\volt}.
\end{align*}

In \autoref{sec:mod} wird beschrieben wie der Modulationsgrad eines solchen frequenzmodulierten Signals bestimmt werden kann. Dazu wurde aus den aufgenommenen Daten der Frequenzhub der Frequenzmodulation bestimmt. Es wurde in der Auswertung aus den Daten ein gesamter Frequenzhub von 
\begin{align*}
 \Delta f &= \SI{260.2}{\kilo\hertz} 
\end{align*}
bestimmt.
Das heißt, dass sich für das hier erzeugte frequenzmodulierte Signal ein Modulationsgrad von 
\begin{equation*}
	m = \frac{\frac{\Delta f}{2}}{f_\text{M}} = \frac{\frac{\SI{260.2}{\kilo\hertz}}{2}}{\SI{200}{\kilo\hertz}} \approx 0.65
\end{equation*}
ergibt.

\subsection{Demodulation modulierter Signale}
Bei der folgenden Messreihe wurde die Schaltung eines phasenempfindlichen Gleichrichters mit einem Ringmodulator aufgebaut, um dessen Eignung zur Demodulation zu untersuchen. Dabei wurden drei verschiedene Ausgangssignale mit unterschiedlichen Phasenverschiebungen gemessen. 
Zu den folgenden Signalen wurde jeweils auch eine Phasenverzögerung bestimmt.
Die erste dieser Messungen mit phasenempfindlichem Gleichrichter ergab folgendes Ausgangssignal:
\begin{figure}[H]
  \centering
  \includegraphics[width=.8\textwidth]{Bilder/osz/scope_6.png}
  \caption{Erste Messung eines phasenverschobenen Ausgangssignals durch phasenempfindlichen Gleichrichter mit Ringmodulator.}
  \label{fig:amplModOszi2}
\end{figure}
\noindent
Die bei der Messung zu \autoref{fig:amplModOszi2} bestimmten Werte sind 
\begin{align*}
  f_\text{M} &= \SI{500}{\kilo\hertz} & U_\text{M} &= \SI{500}{\milli\volt}.
\end{align*}
Zu dieser Messung wurde zusätzlich eine Phasenverzögerung von $T = \SI{250}{\nano\second}$ ermittelt. 
Bei der zweiten Messung mit phasenempfindlichem Gleichrichter konnte das in \autoref{fig:amplModOszi3} zu sehende Signal ermittelt werden.
\begin{figure}[H]
  \centering
  \includegraphics[width=.8\textwidth]{Bilder/osz/scope_7.png}
  \caption{Zweite Messung eines phasenverschobenen Ausgangssignals durch phasenempfindlichen Gleichrichter mit Ringmodulator.}
  \label{fig:amplModOszi3}
\end{figure}
Dabei wurden außerdem die Werte 
\begin{align*}
  f_\text{M} &= \SI{500}{\kilo\hertz} & U_\text{M} &= \SI{500}{\milli\volt}.
\end{align*}
ermittelt.
Hier wurde eine Phasenverzögerung von $T = \SI{290}{\nano\second}$ bestimmt.
Die dritte Messung dieser Messreihe hatte eine deutlich geringere Phasenverzögerung, wie in \autoref{fig:amplModOszi4} zu sehen ist.
\begin{figure}[H]
  \centering
  \includegraphics[width=.8\textwidth]{Bilder/osz/scope_8.png}
  \caption{Dritte Messung eines phasenverschobenen Ausgangssignals, hier mit deutlich geringerer Phasenverzögerung.}
  \label{fig:amplModOszi4}
\end{figure}

Die zu \autoref{fig:amplModOszi4} bestimmten Werte sind 
\begin{align*}
  f_\text{M} &= \SI{500}{\kilo\hertz} & U_\text{M} &= \SI{500}{\milli\volt}.
\end{align*}

Die hier ermittelte Phasenverzögerung beträgt $T = \SI{40}{\nano\second}$.

Als nächste Messreihe wurde die Demodulation eines amplitudenmodulierten Signals mit Trägerabstrahlung über einen Ringmodulator und einen Tiefpass  realisiert. Hinter dem Tiefpass wurde dabei folgendes in \autoref{fig:amplModOszi5} zu sehende Signal aufgenommen.

\begin{figure}[H]
  \centering
  \includegraphics[width=.8\textwidth]{Bilder/osz/scope_9.png}
  \caption{Moduliertes Signal(gelb) und Modulationsschwingung(grün) zur Demodulation des amplitudenmodulierten Signals mit Trägerabstrahlung über einen Ringmodulator und einen Tiefpass.}
  \label{fig:amplModOszi5}
\end{figure}

Die zu dieser Messung bestimmten Werte sind 
\begin{align*}
  f_\text{M} &= \SI{200}{\kilo\hertz} & U_\text{M} &= \SI{500}{\milli\volt}\\
  f_\text{T} &= \SI{5}{\mega\hertz} & U_\text{T} &= \SI{1}{\volt}.
\end{align*}
Wie zu erwarten ist am Ausgang des Tiefpasses eine phasenverschobene Kopie des Modulationssignals zu sehen.


Als nächstes wurde die Demodulation eines amplitudenmodulierten Signals durchgeführt, bei der das amplitudenmodulierte Signal gleichgerichtet und dann über einen Tiefpass geführt wurde.
Dabei wurde hinter dem Gleichrichter folgendes Signal aufgenommen:

\begin{figure}[H]
  \centering
  \includegraphics[width=.8\textwidth]{Bilder/osz/scope_14.png}
  \caption{Nach dem Gleichrichter aufgenommenes Signal bei der Demodulation eines amplitudenmodulierten Signals über einen Gleichrichter und einen Tiefpass.}
  \label{fig:amplModOszi6}
\end{figure}

Die zu \autoref{fig:amplModOszi6} bestimmten Werte sind 
\begin{align*}
  f_\text{M} &= \SI{20}{\kilo\hertz} & U_\text{M} &= \SI{500}{\milli\volt}\\
  f_\text{T} &= \SI{500}{\kilo\hertz} & U_\text{T} &= \SI{1}{\volt}.
\end{align*}

Nach dem Tiefpass konnte folgendes Signal gemessen werden:

\begin{figure}[H]
  \centering
  \includegraphics[width=.8\textwidth]{Bilder/osz/scope_10.png}
  \caption{Nach dem Tiefpass aufgenommenes Signal bei der Demodulation eines amplitudenmodulierten Signals über einen Gleichrichter und einen Tiefpass.}
  \label{fig:amplModOszi10}
\end{figure}

Zu der Messung nach dem Tiefpass (\autoref{fig:amplModOszi10}) wurden die Werte 
\begin{align*}
  f_\text{M} &= \SI{20}{\kilo\hertz} & U_\text{M} &= \SI{500}{\milli\volt}\\
  f_\text{T} &= \SI{500}{\kilo\hertz} & U_\text{T} &= \SI{1}{\volt}
\end{align*}
bestimmt. \newline

\noindent
Abschließend wurde ein frequenzmoduliertes Signal demoduliert. Bei diesem Vorgang wurde das frequenzmodulierte Signal einem LC-Schwingkreis zugeführt und so in ein amplitudenmoduliertes Signal umgewandelt. Schließlich wurde das Signal über einen Gleichrichter und einen Tiefpass demoduliert. Der zeitliche Verlauf des Signals auf dem Oszilloskopschirm wurde jeweils als Bild aufgenommen und ist im Folgenden dargestellt.

Nach dem Schwingkreis wurde folgendes Signal aufgenommen:
\begin{figure}[H]
  \centering
  \includegraphics[width=.8\textwidth]{Bilder/osz/scope_11.png}
  \caption{Signal hinter dem Schwingkreis.}
  \label{fig:amplModOszi7}
\end{figure}

Die zu \autoref{fig:amplModOszi7} bestimmten Werte sind 
\begin{align*}
  f_\text{M} &= \SI{20}{\kilo\hertz} & U_\text{M} &= \SI{500}{\milli\volt}\\
  f_\text{T} &= \SI{500}{\kilo\hertz} & U_\text{T} &= \SI{1}{\volt}.
\end{align*}

Dieses Signal wurde hinter dem Gleichrichter aufgenommen:
\begin{figure}[H]
  \centering
  \includegraphics[width=.8\textwidth]{Bilder/osz/scope_12.png}
  \caption{Nach dem Gleichrichter aufgenommenes Signal.}
  \label{fig:amplModOszi8}
\end{figure}

Die zur Messung nach dem Gleichrichter bestimmten Werte sind 
\begin{align*}
  f_\text{M} &= \SI{20}{\kilo\hertz} & U_\text{M} &= \SI{500}{\milli\volt}\\
  f_\text{T} &= \SI{500}{\kilo\hertz} & U_\text{T} &= \SI{1}{\volt}.
\end{align*}

Wie zu erwarten wurde hinter dem Tiefpass ein phasenverschobenes Signal aufgenommen:
\begin{figure}[H]
  \centering
  \includegraphics[width=.8\textwidth]{Bilder/osz/scope_13.png}
  \caption{Phasenverschobenes Signal hinter dem Tiefpass.}
  \label{fig:amplModOszi9}
\end{figure}

Die zu \autoref{fig:amplModOszi9} bestimmten Werte sind 
\begin{align*}
  f_\text{M} &= \SI{20}{\kilo\hertz} & U_\text{M} &= \SI{500}{\milli\volt}\\
  f_\text{T} &= \SI{500}{\kilo\hertz} & U_\text{T} &= \SI{1}{\volt}.
\end{align*}
\section{Discussion}
The contrast measurement yielded results that are consistent with theoretical expectations. 
The observed offset angle can be attributed either to a constant misalignment of the polarizer or to a systematic offset in the polarization angle of the laser source. 
The amplitude of the contrast function was found to be below its theoretical maximum, 
which indicates a non-ideal alignment of the interferometer setup.

The refractive index of the investigated glass sample was determined experimentally. 
Due to the wide variety of existing glass types and the lack of information regarding the specific composition of the sample, 
a direct comparison with literature values is not meaningful. 
Nevertheless, 
the measured refractive index lies within the typical range reported for common glass materials, 
indicating reasonable agreement with expected values.

The refractive index of the gas present in the laboratory under standard atmospheric conditions was also determined. 
While a small deviation from the literature value for air under standard conditions was observed, 
this difference is considered plausible. 
It can be attributed to variations in the exact gas composition in the room as well as to environmental factors such as humidity, 
which were not explicitly controlled during the measurement.

\newpage
%\input{content/anhang.tex}
%\newpage
\printbibliography{}


\end{document}