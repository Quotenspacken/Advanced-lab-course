\section{Evaluation}
\subsection{Evaluation of the Standard BB84 Protocol}
\label{ch:BB84}

In the first part of the experiment, the BB84 protocol was implemented without the presence of an eavesdropper. 
Alice generated 52 random basis and bit choices and Bob measured the transmitted photons in a randomly chosen base. 
The bases and bits were randomly generated using ChatGPT.
The results were recorded in the Table \ref{tab:versuch1}.
Here, the '+' stands for the rectilinear basis and '×' for the diagonal basis.
\begin{longtable}{c c c c c}
\caption{Measured raw data for Alice and Bob (Standard BB84).} \\
\hline
\# & transmitted bit & Alice Basis & Bob Basis & received bit \\
\hline
\endfirsthead
\hline
\caption{Measured raw data for Alice and Bob (Standard BB84).} \\
\# & transmitted bit & Alice Basis & Bob Basis & received bit \\
\hline
\endhead
\hline
\endfoot
\hline
\endlastfoot
\rowcolor{red!30}
1  & 1 & + & x & 0 \\
2  & 0 & + & + & 0 \\ 
3  & 0 & x & x & 0 \\
4  & 1 & x & x & 1 \\
5  & 0 & + & + & 0 \\
6  & 1 & + & + & 1 \\
7  & 1 & x & x & 1 \\
8  & 0 & x & x & 0 \\
9  & 0 & + & + & 0 \\
\rowcolor{red!30}
10 & 1 & + & x & 1 \\
11 & 1 & x & x & 1 \\
\rowcolor{red!30}
12 & 0 & x & + & 1 \\
13 & 0 & + & + & 0 \\
14 & 1 & + & + & 1 \\
15 & 0 & x & x & 0 \\
\rowcolor{red!30}
16 & 1 & x & + & 0 \\
\rowcolor{red!30}
17 & 0 & + & x & 1 \\
\rowcolor{red!30}
18 & 1 & + & x & 0 \\
19 & 0 & x & x & 0 \\
20 & 1 & + & + & 0 \\
\rowcolor{red!30}
21 & 1 & x & + & 0 \\
\rowcolor{red!30}
22 & 0 & + & x & 1 \\
\rowcolor{red!30}
23 & 0 & + & x & 0 \\
\rowcolor{red!30}
24 & 1 & x & + & 1 \\
25 & 1 & + & + & 1 \\
26 & 1 & x & x & 1 \\
27 & 0 & + & + & 0 \\
\rowcolor{red!30}
28 & 0 & x & + & 0 \\
\rowcolor{red!30}
29 & 1 & + & x & 0 \\
30 & 1 & x & x & 1 \\
\rowcolor{red!30}
31 & 0 & x & + & 0 \\
\rowcolor{red!30}
32 & 0 & + & x & 1 \\
\rowcolor{red!30}
33 & 1 & + & x & 0 \\
\rowcolor{red!30}
34 & 1 & x & + & 0 \\
35 & 0 & + & + & 0 \\
36 & 0 & x & x & 0 \\
37 & 1 & + & + & 1 \\
38 & 1 & x & x & 1 \\
\rowcolor{red!30}
39 & 0 & + & x & 0 \\
\rowcolor{red!30}
40 & 0 & x & + & 0 \\
41 & 1 & + & + & 1 \\
\rowcolor{red!30}
42 & 0 & + & x & 0 \\
43 & 1 & x & x & 1 \\
\rowcolor{red!30}
44 & 0 & x & + & 1 \\
\rowcolor{red!30}
45 & 1 & + & x & 0 \\
46 & 0 & + & + & 0 \\
47 & 1 & x & x & 1 \\
\rowcolor{red!30}
48 & 0 & x & + & 1 \\
49 & 0 & + & + & 0 \\
\rowcolor{red!30}
50 & 1 & + & x & 0 \\
\rowcolor{red!30}
51 & 1 & x & + & 1 \\
\rowcolor{red!30}
52 & 0 & + & x & 0 \\  
\hline
\label{tab:versuch1}
\end{longtable}
Highlighted in red are the transmitted bits that did not have the same basis.
Sent and received bits that were measured in different bases agree in only $36\%$  of cases. 

After basis reconciliation, only the events where both parties used the same basis were retained (Sifting). 
From this subset, the \textbf{Quantum Bit Error Rate (QBER)} was calculated as
\begin{equation}
    QBER = \frac{N_{\text{errors}}}{N_{\text{sifted}}} \times 100\%,
    \label{eq:QBER}
\end{equation}
where \( N_{\text{errors}} \) is the number of bits that differ between Alice and Bob, and \( N_{\text{sifted}} \) is the total number of bits after sifting.
The result is shown in Table \ref{tab:result1}.
\begin{table}[ht]
\centering
\caption{Results of basis reconciliation and QBER calculation (without Eve).}
\begin{tabular}{l c}
\hline
Total number of raw bits & 52 \\
Number of matched bases \( N_{\text{sifted}} \) & 25 \\
Number of errors \( N_{\text{errors}} \) &  0\\
QBER &  0\\
\hline
\label{tab:result1}
\end{tabular}
\end{table}
A QBER close to zero indicates a stable and noise-free quantum channel, confirming correct alignment and low detector noise.
This is also a clear sign that there is no eavesdropper.

In the next step, 
Eve's detection module was placed between Alice and Bob to simulate an eavesdropping attack.
Eve measured each photon in a random basis and re-sent the measured bit in the corresponding polarization to Bob. 
The data was again recorded and analyzed in the same way.
Table \ref{tab:versuch2} shows this transfer of 52 bits using the Eve module.
\begin{longtable}{c c c c c c}
\caption{Measured data for Alice and Bob with Eve (Standard BB84).} \\
\hline
\# & transmitted bit & Alice Basis & Eve Basis & Bob Basis & received bit \\
\hline
\endfirsthead
\hline
\caption{Measured raw data for Alice and Bob (Standard BB84).} \\
\# & transmitted bit & Alice Basis & Eve Basis & Bob Basis & received bit \\
\hline
\endhead
\hline
\endfoot
\hline
\endlastfoot
1  & 1 & x & + & x & 1 \\
\rowcolor{green}
2  & 0 & + & x & + & 1 \\
\rowcolor{red!30}
3  & 1 & + & + & x & 0 \\
\rowcolor{red!30}
4  & 0 & x & + & + & 0 \\
5  & 1 & x & x & x & 1 \\
\rowcolor{red!30}
6  & 0 & + & x & + & 1 \\
7  & 1 & + & + & + & 1 \\
8  & 1 & x & + & x & 1 \\
\rowcolor{red!30}
9  & 0 & x & x & + & 1 \\
\rowcolor{green}
10 & 1 & x & + & x & 0 \\
\rowcolor{green}
11 & 1 & + & x & + & 0 \\
\rowcolor{green}
12 & 1 & x & + & x & 0 \\
\rowcolor{red!30}
13 & 0 & + & x & x & 1 \\
\rowcolor{red!30}
14 & 0 & x & x & + & 0 \\
15 & 1 & + & + & + & 1 \\
16 & 1 & x & x & x & 1 \\
\rowcolor{green}
17 & 0 & x & + & x & 1 \\
18 & 1 & + & x & + & 1 \\
19 & 1 & x & x & x & 1 \\
20 & 0 & + & + & + & 0 \\
\rowcolor{red!30}
21 & 0 & x & x & + & 1 \\
\rowcolor{red!30}
22 & 1 & + & + & x & 1 \\
23 & 1 & x & x & x & 1 \\
24 & 1 & + & + & + & 1 \\
\rowcolor{red!30}
25 & 0 & + & x & x & 1 \\
\rowcolor{red!30}
26 & 1 & x & + & + & 1 \\
\rowcolor{red!30}
27 & 1 & x & + & + & 0 \\
28 & 0 & x & x & x & 0 \\
\rowcolor{red!30}
29 & 0 & + & x & x & 1 \\
30 & 1 & + & + & + & 1 \\
31 & 1 & x & + & x & 1 \\
32 & 1 & + & x & + & 1 \\
\rowcolor{red!30}
33 & 0 & x & + & + & 1 \\
34 & 1 & x & + & x & 1 \\
\rowcolor{red!30}
35 & 0 & + & x & x & 0 \\
36 & 1 & + & + & + & 1 \\
\rowcolor{red!30}
37 & 1 & x & x & + & 0 \\
38 & 1 & x & + & x & 1 \\
39 & 0 & x & x & x & 0 \\
40 & 0 & + & x & + & 0 \\
\rowcolor{red!30}
41 & 1 & + & + & x & 0 \\
\rowcolor{red!30}
42 & 0 & x & x & + & 1 \\
\rowcolor{red!30}
43 & 1 & x & + & + & 0 \\
\rowcolor{red!30}
44 & 0 & + & x & x & 1 \\
\rowcolor{red!30}
45 & 1 & + & x & x & 0 \\
\rowcolor{red!30}
46 & 0 & x & + & + & 1 \\
47 & 1 & x & + & x & 1 \\
48 & 0 & + & x & + & 0 \\
\rowcolor{red!30}
49 & 1 & + & + & x & 1 \\
\rowcolor{red!30}
50 & 1 & x & x & + & 0 \\
51 & 0 & x & x & x & 0 \\
52 & 0 & + & + & + & 0 \\
\hline
\label{tab:versuch2}
\end{longtable}
The rows where Alice and Bob did not choose the same basis are again highlighted in red.
This time, 
the transmissions where Alice and Bob chose the same basis but have different bits are additionally highlighted in green.
This can occurs whenever Alice and Eve have different bases.
It is important to note, 
that even when Alice and Eve have different bases, 
it can still randomly happen that Alice and Bob end up with the same bit.
Calculating the \textbf{QBER} again yields the results shown in Table \ref{tab:result2}.
\begin{table}[ht]
\centering
\caption{Results of BB84 protocol with Eve inserted.}
\begin{tabular}{l c}
\hline
Total number of raw bits & 52 \\
Number of matched bases \( N_{\text{sifted}} \) & 28  \\
Number of errors \( N_{\text{errors}} \) &   5 \\
QBER (with Eve) & 17\% \\
\hline
\label{tab:result2}
\end{tabular}
\end{table}
As expected, 
the QBER significantly increased (typically around 25\%) due to Eve’s random measurement bases disturbing the photon states. 
This demonstrates the fundamental principle of BB84: any eavesdropping attempt introduces detectable errors.





\subsection{Message Transmissions}
As an example, 
using the transmissions from Chapter~\ref{ch:BB84} without the presence of Eve, 
we can transmit a encrypoted message.
The example message is \textit{bear}.
The binary representations of the required letters can be found in Table~\ref{tab:bin}.
\begin{table}[ht]
\centering
\caption{Binary representation of the alphabet.}
\begin{tabular}{c c c c c c}
\hline
A & 0 & 0 & 0 & 0 & 0 \\
B & 0 & 0 & 0 & 0 & 1 \\
E & 0 & 0 & 1 & 0 & 0 \\
R & 1 & 0 & 0 & 0 & 1 \\
\hline
\label{tab:bin}
\end{tabular}
\end{table}
So the Binary representation (Data Bit) of the word is
\begin{equation*}
    00001 00000 00100 10001.
\end{equation*}
Therefore, 
20 key bits are required to decrypt the message.
The first 20 sifted bits from Table~\ref{tab:versuch1} are used:
\begin{equation*}
    00101 10010 10001 10100.    
\end{equation*}
The Data and Key Bit are added (bitwise) and the reult is the Encrypoted word.
\begin{align*}
     &00001 00000 00100 10001 \\
    +&00101 10010 10001 10100 \\ 
    =&00000 10010 10101 00101
\end{align*}
The encrypoted word now is \textit{ASVF}.





\subsection{Evaluation of the Decoy State Extension}
In the second part, the decoy state version of the protocol was implemented using two wavelengths (\textcolor{green}{green} and \textcolor{red}{red} lasers). 
Alice randomly selected among different polarization states corresponding to signal and decoy states, 
as defined in Table~\ref{tab:prot}. 
\begin{table}[ht]
\centering
\caption{Ternary protokcol encoding chart summary for two wavelengths.}
\begin{tabular}{c c c c c}
\hline
 Basis & trit & State & Bit & Decoy state \\
\hline
+ & 0 & \textcolor{red}{0°}     & 0 &                       \\
+ & 1 & \textcolor{red}{90°}    & 1 &                       \\     
+ & 2 & \textcolor{green}{0°}   & 0 &                       \\ 
+ & d &                         &   & \textcolor{green}{90°}\\         
x & 0 & \textcolor{green}{-45°} & 0 &                       \\     
x & 1 & \textcolor{green}{45°}  & 1 &                       \\      
x & 2 & \textcolor{red}{-45°}   & 0 &                       \\     
x & d &                         &   &  \textcolor{red}{45°} \\       
\hline
\label{tab:prot}
\end{tabular}
\end{table}

First, 52 bits are exchanged again between Alice and Bob without Eve.
The data are shown in the table \ref{tab:versuch3}
\begin{longtable}{c c c c c}
\caption{Measured raw data for Alice and Bob with decoy state.} \\
\hline
\# & transmitted Trit & Alice Basis & Bob Basis & received Trit \\
\hline
\endfirsthead
\hline
\caption{Measured raw data for Alice and Bob (Standard BB84).} \\
\# & transmitted Trit & Alice Basis & Bob Basis & received Trit \\
\hline
\endhead
\hline
\endfoot
\hline
\endlastfoot
1  & 2 & x & x & 2 \\
2  & 1 & + & + & 1 \\
\rowcolor{red!30}
3  & 2 & + & x & 0 \\
\rowcolor{red!30}
4  & 0 & x & + & 2 \\
\rowcolor{red!30}
5  & 1 & + & x & 1 \\
6  & 0 & + & + & 0 \\
7  & 0 & x & x & 0 \\
8  & 2 & + & + & 2 \\
9  & 1 & x & x & 1 \\
10 & 2 & x & x & 2 \\
11 & 1 & + & + & 1 \\
\rowcolor{red!30}
12 & 0 & + & x & 0 \\
\rowcolor{red!30}
13 & 0 & x & + & 2 \\
14 & 0 & + & + & 0 \\
\rowcolor{red!30}
15 & 1 & + & x & 2 \\
16 & 1 & x & x & 1 \\
\rowcolor{red!30}
17 & 2 & x & + & 1 \\
\rowcolor{red!30}
18 & 2 & + & x & 1 \\
19 & 0 & x & x & 0 \\
20 & 1 & + & + & 1 \\
21 & 0 & + & + & 0 \\
22 & 1 & x & x & 1 \\
\rowcolor{red!30}
23 & 0 & x & + & 1 \\
\rowcolor{red!30}
24 & 1 & + & x & 2 \\
\rowcolor{red!30}
25 & 0 & x & + & 2 \\
26 & 2 & x & x & 2 \\
27 & 1 & x & x & 1 \\
28 & 0 & + & + & 0 \\
29 & 1 & + & + & 1 \\
30 & 0 & x & x & 0 \\
31 & 0 & + & + & 0 \\
32 & 1 & x & x & 1 \\
\rowcolor{red!30}
33 & 0 & x & + & 1 \\
\rowcolor{red!30}
34 & 1 & + & x & 1 \\
35 & 2 & + & + & 2 \\
36 & 1 & x & x & 1 \\
\rowcolor{red!30}
37 & 1 & + & x & 2 \\
\rowcolor{red!30}
38 & 1 & x & + & 1 \\
39 & 2 & + & + & 2 \\
40 & 0 & x & x & 0 \\
41 & 1 & + & + & 1 \\
\rowcolor{red!30}
42 & 2 & + & x & 0 \\
43 & 2 & x & x & 2 \\
\rowcolor{red!30}
44 & 1 & x & + & 2 \\
45 & 1 & + & + & 1 \\
\rowcolor{red!30}
46 & 2 & + & x & 0 \\
\rowcolor{red!30}
47 & 0 & x & + & 2 \\
\rowcolor{red!30}
48 & 1 & + & x & 2 \\
\rowcolor{red!30}
49 & 0 & + & x & 1 \\
\rowcolor{red!30}
50 & 0 & x & + & 1 \\
\rowcolor{red!30}
51 & 1 & x & + & 1 \\
\rowcolor{red!30}
52 & 2 & + & x & 0 \\
\hline
\label{tab:versuch3}
\end{longtable}

After sifting, the Quantum Bit Error Rate (\refeq{eq:QBER}) and the fraction of detected decoy states 
\begin{equation}
    f_{\text{decoy}} = \frac{N_{\text{decoys}}}{N_{\text{matched}}},
    \label{eq:decoy}
\end{equation}
where calculated and are shown in \ref{tab:result3}.
\begin{table}[ht]
\centering
\caption{Results for BB84 with decoy states (without Eve).}
\begin{tabular}{l c}
\hline
Number of matched bases \( N_{\text{matched}} \) &  28\\
Number of errors \( N_{\text{errors}} \) & 0 \\
Number of decoy detections \( N_{\text{decoys}} \) & 0 \\
QBER &  0\%\\
Decoy fraction & 0\% \\
\hline
\label{tab:result3}
\end{tabular}
\end{table}
In this case as well, 
the QBER and decoy fraction indicate that no eavesdropper is present.

Once again, 
the Eve module was reintroduced, 
and the measurements were repeated using the same decoy state as before.
The transmitted data are shown in Table~\ref{tab:versuch4}.
First, 52 bits were exchanged between Alice and Bob without Eve.
The corresponding data are shown in Table~\ref{tab:versuch3}.
\begin{longtable}{c c c c c c}
\caption{Measured raw data for Alice and Bob with decoy state.} \\
\hline
\# & transmitted Trit & Alice Basis & Eve Basis & Bob Basis & received Trit \\
\hline
\endfirsthead
\hline
\caption{Measured raw data for Alice and Bob (Standard BB84).} \\
\# & transmitted Trit & Alice Basis & Eve Basis & Bob Basis & received Trit \\
\hline
\endhead
\hline
\endfoot
\hline
\endlastfoot
1  & 2 & + & x & + & 2 \\
\rowcolor{red!30}
2  & 1 & + & + & x & 2 \\
3  & 1 & x & x & x & 1 \\
\rowcolor{red!30}
4  & 2 & x & + & + & 0 \\
5  & 2 & + & x & + & 2 \\
\rowcolor{yellow}
6  & 2 & x & + & x & d \\
\rowcolor{green}
7  & 1 & x & + & x & 0 \\
8  & 0 & + & x & + & 0 \\
9  & 0 & x & + & x & 0 \\
\rowcolor{green}
10 & 1 & + & x & + & 0 \\
\rowcolor{red!30}
11 & 2 & + & x & x & 0 \\
12 & 2 & x & + & x & 2 \\
\rowcolor{red!30}
13 & 1 & x & x & + & d \\
14 & 0 & + & + & + & 0 \\
15 & 0 & x & x & x & 0 \\
\rowcolor{red!30}
16 & 1 & + & + & x & 0 \\
\rowcolor{yellow}
17 & 2 & + & x & + & d \\
18 & 2 & x & x & x & 2 \\
\rowcolor{green}
19 & 1 & x & + & x & 0 \\
20 & 0 & + & + & + & 0 \\
\rowcolor{red!30}
21 & 0 & x & x & + & d \\
\rowcolor{red!30}
22 & 1 & + & + & x & 2 \\
23 & 2 & + & x & + & 2 \\
24 & 2 & x & + & x & 2 \\
\rowcolor{red!30}
25 & 1 & x & x & + & d \\
\rowcolor{red!30}
26 & 0 & + & x & x & d \\
27 & 0 & x & + & x & 0 \\
\rowcolor{green}
28 & 1 & + & x & + & 0 \\
\rowcolor{red!30}
29 & 2 & + & + & x & 1 \\
\rowcolor{red!30}
30 & 2 & x & x & + & 1 \\
31 & 1 & x & x & x & 1 \\
\rowcolor{red!30}
32 & 0 & + & + & x & 2 \\
\rowcolor{red!30}
33 & 2 & x & + & + & 0 \\
34 & 1 & + & x & + & 1 \\
\rowcolor{red!30}
35 & 2 & + & + & x & 1 \\
\rowcolor{green}
36 & 1 & x & x & x & 0 \\
\rowcolor{red!30}
37 & 2 & x & x & + & 0 \\
38 & 2 & + & + & + & 2 \\
\rowcolor{red!30}
39 & 0 & x & x & + & 2 \\
\rowcolor{yellow}
40 & 2 & + & x & + & d \\
\rowcolor{red!30}
41 & 2 & + & + & x & 0 \\
42 & 2 & x & x & x & 2 \\
\rowcolor{red!30}
43 & 2 & x & + & + & 1 \\
44 & 2 & + & x & + & 2 \\
\rowcolor{yellow}
45 & 2 & x & + & x & d \\
\rowcolor{green}
46 & 1 & + & x & + & 0 \\
\rowcolor{red!30}
47 & 2 & + & + & x & 0 \\
48 & 2 & x & x & x & 2 \\
\rowcolor{red!30}
49 & 1 & x & + & + & 2 \\
\rowcolor{red!30}
50 & 2 & + & x & x & 1 \\
\rowcolor{red!30}
51 & 2 & x & x & + & 0 \\
\rowcolor{red!30}
52 & 1 & + & + & x & 2 \\
\hline
\label{tab:versuch4}
\end{longtable}
Here, 
the transmissions with different bases between Alice and Bob are marked in red, 
and a different bit between the two when the basis is the same is marked in green.
Additionally, 
all decoy states with the same basis are colored yellow.
Using equations \eqref{eq:QBER} and \eqref{eq:decoy}, 
it is possible to calculate the results presented in Table~\ref{tab:result4}.
\begin{table}[ht]
\centering
\caption{Results for BB84 with decoy states (without Eve).}
\begin{tabular}{l c}
\hline
Number of matched bases \( N_{\text{matched}} \) &  28\\
Number of errors \( N_{\text{errors}} \) & 6 \\
Number of decoy detections \( N_{\text{decoys}} \) & 4 \\
QBER &  21\%\\
Decoy fraction & 14\% \\
\hline
\label{tab:result4}
\end{tabular}
\end{table}
As expected, 
the \textbf{QBER} and the decoy fraction lie significantly above 0\%.
This indicates the presence of an eavesdropper, 
since whenever the basis between Alice and Eve differs, 
Eve has only a 50\% chance of detecting and transmitting the correct bit to Bob.




